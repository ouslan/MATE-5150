\documentclass[10pt, oneside]{article}
\usepackage{amsmath, amsthm, amssymb, calrsfs, wasysym, verbatim, bbm, color, graphics, geometry}

\geometry{tmargin=.75in, bmargin=.75in, lmargin=.75in, rmargin = .75in}

\newcommand{\R}{\mathbb{R}}
\newcommand{\C}{\mathbb{C}}
\newcommand{\Z}{\mathbb{Z}}
\newcommand{\N}{\mathbb{N}}
\newcommand{\Q}{\mathbb{Q}}
\newcommand{\Cdot}{\boldsymbol{\cdot}}

\newtheorem{thm}{Theorem}
\newtheorem{defn}{Definition}
\newtheorem{conv}{Convention}
\newtheorem{rem}{Remark}
\newtheorem{lem}{Lemma}
\newtheorem{cor}{Corollary}


\title{MATE 5150: Sistemas de Ecuaciones Lineales (Repasos)}
\author{Alejandro Ouslan}
\date{Ano Academico 2024-2025}

\begin{document}

\maketitle
\tableofcontents

\vspace{.25in}

\section{Matrices}

\begin{defn}[Matrices]
	Sean $m$ y $n$ enteros positivos. Una matriz de orden $m \times n$ es un arreglo rectangular de
	$m$ elementos dispuestos en $m$ filas y $n$ columnas como sigue:
	\[
		A = \begin{bmatrix}
			a_{11} & a_{12} & \cdots & a_{1n} \\
			a_{21} & a_{22} & \cdots & a_{2n} \\
			\vdots & \vdots & \ddots & \vdots \\
			a_{m1} & a_{m2} & \cdots & a_{mn}
		\end{bmatrix}
	\]
\end{defn}
\subsection{Notation}
\begin{itemize}
	\item Letras mayúsculas para matrices y minúsculas para sus elementos
	\item Para la matriz $A_{m \times n}$ $i$ representa las filas $i \in \{1,2,\ldots, m\}$ y $j$ representa
	      las columnas $j \in \{1,2,\ldots, n\}$
	      \begin{itemize}
		      \item $A_{1j} = \begin{bmatrix} a_1j \\ a_2j \\ \vdots \\ a_{2j}  \end{bmatrix}$
		      \item $A_{i1} = \begin{bmatrix} a_m1 && a_m2 && \ldots && a_{mj}  \end{bmatrix}$
	      \end{itemize}
	\item Escribimos $A = \begin{bmatrix} a_{ij} \end{bmatrix}$ de forma reducida de la matriz $A$.
	\item $\mathbb{R}_{m \times n}$ representa al conjunto de todas las matrices de orden $m \times n$ de entrada real y
	      $\mathbb{C}_{m \times n}$ de entradas Complejas.
\end{itemize}

\end{document}
