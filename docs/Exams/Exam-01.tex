\documentclass[10pt, oneside]{article}
\usepackage{amsmath, amsthm, amssymb, calrsfs, wasysym, verbatim, bbm, color, graphics, geometry}

\geometry{tmargin=.75in, bmargin=.75in, lmargin=.75in, rmargin = .75in}

\newcommand{\R}{\mathbb{R}}
\newcommand{\C}{\mathbb{C}}
\newcommand{\Z}{\mathbb{Z}}
\newcommand{\N}{\mathbb{N}}
\newcommand{\Q}{\mathbb{Q}}
\newcommand{\Cdot}{\boldsymbol{\cdot}}

\newtheorem{thm}{Theorem}
\newtheorem{defn}{Definition}
\newtheorem{conv}{Convention}
\newtheorem{rem}{Remark}
\newtheorem{lem}{Lemma}
\newtheorem{cor}{Corollary}


\title{MATE 5150: Exam 01 Review}
\author{Alejandro Ouslan}
\date{Academic Year 2024-2025}

\begin{document}

\maketitle
\tableofcontents

\vspace{.25in}

\section{Determine if $T$ is a Vector Space}

\subsection{Conditions}
\begin{enumerate}
	\item $x + y  = y + x$
	\item $x + (y + z) = (x + y) + z$
	\item $\exists 0 \in V \quad \text{s.t.} \quad x + 0 = x$
	\item $\exists 0 \in V \quad \text{s.t.} \quad x + y = 0$
	\item $1x = x$
	\item $a(bx) = (ab)x$
	\item $a(x + y) = ax + ay$
	\item $(a + b)x = ax + bx$
\end{enumerate}

\subsection{Summery of the Conditions}
\begin{enumerate}
	\item $0x = 0$
	\item $(-a)x = -ax = a(-x)$
	\item $a0 = 0$
\end{enumerate}
\subsection{Example}

Let $S = \{(a_1, a_2): a_1, a_2 \in \R\}$ For $(a_1, a_2), (b_1, b_2) \in S$ and $c \in \R$, define
\[
	(a_1, a_2) + (b_1, b_2) = (a_1 + b_1, a_2 - b_2) \quad \text{and} \quad c(a_1, a_2) = (ca_1, ca_2)
\]
Property 1: (fails to be a vector space)
\[
	\begin{split}
		(a_1, a_2) + (b_1, b_2) &= (b_1, b_2) + (a_1, a_2) \\
		(a_1 + b_1, a_2 - b_2) &\neq (b_1 + a_1, b_2 - a_2) \\
	\end{split}
\]
\section{Determine if $T$ is in the Span of $S$}

\section{Proove that the pare of vectors is a basis for vector space $V$}

\begin{enumerate}
	\item Proove that the set $S$ is linearly independent.
	\item Proove that the set $S$ spans $V$.
\end{enumerate}

\section{Generate a polynomial of Lagrange Interpolation of degree $3$}
\[
	P(x) = \sum_{i=0}^{n} y_i \mathcal{L}_i(x)
\]
\[
	\mathcal{L}_i(x) =
	\prod_{\substack{0 \leq j \leq n \\ j \neq i}} \frac{x - x_j}{x_i - x_j} =
	\frac{(x-x_0)\cdots(x-x_i)\cdots(x-x_n)}{(x_i-x_0)\cdots(x_i-x_{i-1})\cdots(x_i-x_n)}
\]

\subsection{Example for $n = 3$}

$$ \{(x_0, f(x_0)), (x_1, f(x_1)), (x_2, f(x_2)), (x_3, f(x_3))\} $$

\[
	\begin{split}
		P(x) &= \frac{(x-x_1)(x-x_2)(x-x_3)}{(x_0-x_1)(x_0-x_2)(x_0-x_3)}f(x_0) + \\
		&+ \frac{(x-x_0)(x-x_2)(x-x_3)}{(x_1-x_0)(x_1-x_2)(x_1-x_3)}f(x_1) + \\
		&+ \frac{(x-x_0)(x-x_1)(x-x_3)}{(x_2-x_0)(x_2-x_1)(x_2-x_3)}f(x_2) + \\
	\end{split}
\]

\section{Determine if $T$ is bijective}

\begin{enumerate}
	\item Proove that $T$ is linear.
	\item Find the kernel of $T$.
	\item Find the Rank of $T$.
	\item Determine if it is 1-1.
\end{enumerate}

\section{Change of Basis}

\section{Answer of Given Questions}

In $R^2$, let $L$ be the line $y = mx$, where $m \neq 0$. Find an expression for $T(x, y)$, where
\begin{enumerate}
	\item $T$ is the reflection of $R^2$ about $L$.
	      \[
		      \begin{split}
			      M_x = \frac{x_1 + x_2}{2} \quad \text{and} \quad M_y = \frac{y_1 + y_2}{2} \\
			      AB = \frac{y_2 - y_1}{x_2 - x_1} = -1/m
		      \end{split}
	      \]
	\item $T$ is the projection on $L$ along the line perpe qqndicular to $L$. (See the definition of projection in the exercises of Section 2.1.)
\end{enumerate}

\end{document}
