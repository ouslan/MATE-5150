\documentclass[10pt, oneside]{article}
\usepackage{amsmath, amsthm, amssymb, calrsfs, wasysym, verbatim, bbm, color, graphics, geometry}

\geometry{tmargin=.75in, bmargin=.75in, lmargin=.75in, rmargin = .75in}

\newcommand{\R}{\mathbb{R}}
\newcommand{\C}{\mathbb{C}}
\newcommand{\Z}{\mathbb{Z}}
\newcommand{\N}{\mathbb{N}}
\newcommand{\Q}{\mathbb{Q}}
\newcommand{\Cdot}{\boldsymbol{\cdot}}

\newtheorem{thm}{Theorem}
\newtheorem{defn}{Definition}
\newtheorem{conv}{Convention}
\newtheorem{rem}{Remark}
\newtheorem{lem}{Lemma}
\newtheorem{cor}{Corollary}


\title{MATE 5150: Exam 01 Review}
\author{Alejandro Ouslan}
\date{Academic Year 2024-2025}

\begin{document}

\maketitle
\tableofcontents

\vspace{.25in}

\section{Determine if $T$ is a Vector Space}

\subsection{Conditions}
\begin{enumerate}
	\item $x + y  = y + x$
	\item $x + (y + z) = (x + y) + z$
	\item $\exists 0 \in V \quad \text{s.t.} \quad x + 0 = x$
	\item $\exists 0 \in V \quad \text{s.t.} \quad x + y = 0$
	\item $1x = x$
	\item $a(bx) = (ab)x$
	\item $a(x + y) = ax + ay$
	\item $(a + b)x = ax + bx$
\end{enumerate}

\subsection{Summery of the Conditions}
\begin{enumerate}
	\item $0x = 0$
	\item $(-a)x = -ax = a(-x)$
	\item $a0 = 0$
\end{enumerate}
\subsection{Example}

Let $S = \{(a_1, a_2): a_1, a_2 \in \R\}$ For $(a_1, a_2), (b_1, b_2) \in S$ and $c \in \R$, define
\[
	(a_1, a_2) + (b_1, b_2) = (a_1 + b_1, a_2 - b_2) \quad \text{and} \quad c(a_1, a_2) = (ca_1, ca_2)
\]
Property 1: (fails to be a vector space)
\[
	\begin{split}
		(a_1, a_2) + (b_1, b_2) &= (b_1, b_2) + (a_1, a_2) \\
		(a_1 + b_1, a_2 - b_2) &\neq (b_1 + a_1, b_2 - a_2) \\
	\end{split}
\]
\section{Determine if $T$ is in the Span of $S$}

\subsection{Definition}

\begin{defn}
	The \textbf{Span} of the set $S$ denoted by $\text{Span}(S)$, is the smallest subspace of $V$
	that contains $S$. That is,
	\begin{itemize}
		\item $\text{Span}(S)$ is a subspace of $V$
		\item For any subspaces $W \subseteq V$ such that $S \subseteq W \implies \text{Span}(S) \subset W$
	\end{itemize}
\end{defn}

\begin{defn}
	Let $S$ be a subset of a vector space $V$.
	\begin{itemize}
		\item If $S = \{v_1, v_2, \ldots, v_n\}$, then $\text{Span}(S)$ is the set of all combinations
		      $r_1, r_2, \ldots, r_n \in \R$
		\item If $S$ is an infinite set then $\text{Span}(S)$ is the set of all linear combinations
		      $r_1v_1 + r_2v_2 + \ldots + r_nv_n$ where $v_1, v_2, \ldots, v_n \in S$ and $r_1, r_2, \ldots, r_n \in \R$, $n \ge 1$
	\end{itemize}
\end{defn}

\subsection{Example 1}

Let $v_1 = (1, 2, 0)$, $v_2 = (3,1,1)$, and $w = (4,-7,3)$. Determine whether $w$ belongs to $\text{Span}(v_1, v_2)$.
\[
	\begin{split}
		w &= r_1v_1 + r_2v_2 \\
		(4,-7,3) &= r_1(1,2,0) + r_2(3,1,1) \\
		(4,-7,3) &= (r_1 + 3r_2, 2r_1 + r_2, r_2) \\
	\end{split}
\]
\begin{center}
	\[
		\begin{split}
			\begin{cases}
				4  & = r_1 + 3r_2 \\
				-7 & = 2r_1 + r_2 \\
				3  & = 0r_1 + r_2 \\
			\end{cases} \\
		\end{split}
	\]
\end{center}
Thus $w = -5v_1 + 3v_2 \in \text{Span}(v_1, v_2)$

\subsection{Example 2}

Let $v_1 = (2,5)$, $v_2 = (1,3)$, show that $\{v_1, v_2\}$ is a $\text{Span}$ for $\R^2$.

Take any vector $W = (a,b) \in \R^2$. We have to chack thatg there exist $r_1, r_2 \in \R$ such that
\[
	w = r_1v_1 + r_2v_2 \iff \begin{cases} a = 2r_1 + r_2 \\ b = 5r_1 + 3r_2 \end{cases}
\]

Coeficients matrix = $\begin{bmatrix} 2 & 1 \\ 5 & 3 \end{bmatrix}$, $det = 1 \neq 0$. Since the matrix
is invertible, the system hasx a unique solution for any $a$ and $b$. Thus $\{v_1, v_2\}$ is a $\text{Span}$ for $\R^2$.

\subsection{Example 2 (Alternative)}
Same as before,

First let us show that vectors $e_1 = (1,0)$ and $e_2 = (0,1)$ are a $\text{Span}(v_1, v_2)$.

\[
	\begin{split}
		e_1 &= 2v_1 + v_2 \iff \begin{cases} 1 = 2 r_1 + r_2 \\ 0 = 5r_1 + 3r_2 \end{cases}
		\iff \begin{cases} r_1 = 3 \\ r_2 = -5 \end{cases} \\
		e_2 &= v_1 + 3v_2 \iff \begin{cases} 0 = 2r_1 + 3r_2 \\ 1 = 5r_1 + 9r_2 \end{cases}
		\iff \begin{cases} r_1 = -1 \\ r_2 = 2 \end{cases}
	\end{split}
\]

Thus $e_1 = 3v_1 - 5v_2$ and $e_2 = -v_1 + 2v_2$. Then for any $w = (a,b) \in \R^2$ we have
\[
	w = ae_1 + be_2 = a(3v_1 - 5v_2) + b(-v_1 + 2v_2) = (3a - b)v_1 + (-5a + 2b)v_2
\]

\section{Proove that the pair of vectors is a basis for vector space $V$}
\begin{defn}
	Let $V$ be a vector space. A  linearly independent set spanning set for $V$ is called a \textbf{basis}.
\end{defn}

\begin{defn}
	A set of vectors $S = \{v_1, v_2, \ldots, v_n\}$ is \textbf{linearly independent} if the only solution to the equation
	\[
		r_1v_1 + r_2v_2 + \ldots + r_nv_n = 0
	\]
	is $r_1 = r_2 = \ldots = r_n = 0$.
\end{defn}
\begin{enumerate}
	\item Proove that the set $S$ is linearly independent.
	\item Proove that the set $S$ spans $V$.
\end{enumerate}

\subsection{Example}
Determine which of the following sets of vectors is a basis for $\R^3$.

\begin{enumerate}
	\item $S_1 = \{(2,-4,1), (0,3,-1), (6,0,-1)\}$
	      \[
		      \begin{split}
			      r_1(2,-4,1) + r_2(0,3,-1) + r_3(6,0,-1) &= (0,0,0) \\
			      \begin{cases}
				      2r_1 + 0r_2 + 6r_3  & = 0 \\
				      -4r_1 + 3r_2 + 0r_3 & = 0 \\
				      r_1 - r_2 - r_3     & = 0
			      \end{cases}
			      \implies \begin{cases}
				      r_1 = -3r_3 \\
				      r_2 = -4r_3 \\
				      0 = -3r_3 - 4r_3 - r_3
			      \end{cases}
		      \end{split}
	      \]
	      Thus $S_1$ is not linearly independent and therefore not a basis for $\R^3$, given
	      that $0 = -3r_1 - 4r_2 + r_3$
	\item $S_2 = \{(1,-3,-2), (-3,1,3), (-2,-10.-2)\}$
\end{enumerate}

\section{Generate a polynomial of Lagrange Interpolation of degree $3$}
\[
	P(x) = \sum_{i=0}^{n} y_i \mathcal{L}_i(x)
\]
\[
	\mathcal{L}_i(x) =
	\prod_{\substack{0 \leq j \leq n \\ j \neq i}} \frac{x - x_j}{x_i - x_j} =
	\frac{(x-x_0)\cdots(x-x_i)\cdots(x-x_n)}{(x_i-x_0)\cdots(x_i-x_{i-1})\cdots(x_i-x_n)}
\]

\subsection{Example for $n = 3$}

$$ \{(x_0, f(x_0)), (x_1, f(x_1)), (x_2, f(x_2)), (x_3, f(x_3))\} $$

\[
	\begin{split}
		P(x) &= \frac{(x-x_1)(x-x_2)(x-x_3)}{(x_0-x_1)(x_0-x_2)(x_0-x_3)}f(x_0) + \\
		&+ \frac{(x-x_0)(x-x_2)(x-x_3)}{(x_1-x_0)(x_1-x_2)(x_1-x_3)}f(x_1) + \\
		&+ \frac{(x-x_0)(x-x_1)(x-x_3)}{(x_2-x_0)(x_2-x_1)(x_2-x_3)}f(x_2) + \\
	\end{split}
\]

\section{Determine if $T$ is bijective}

\begin{enumerate}
	\item Proove that $T$ is linear.
	\item Find the kernel of $T$.
	\item Find the Rank of $T$.
	\item Determine if it is 1-1.
\end{enumerate}

\section{Change of Basis}

The strategy is to find the change of basis matrix $P$ such that $[v]_{\beta} = P[v]_{\alpha}$.

\subsection{Example}
Let $\beta = \{1+ x - x^2, x + x^2, -x + 3x^2\}$ and $\alpha = \{1 + x + x^2, 1 - 2x^2, 4x\}$. and
let $p(x) \in P_2$ be such that $[p(x)]_{\alpha} = \begin{bmatrix} 3 \\ 2 \\ 1 \end{bmatrix}$. Find
$[p(x)]_{\beta}$.
\[
	\begin{split}
		[p(x)]_{\alpha} &= 3(1 + x + x^2) + 2(1 - 2x^2) + 1(4x) \\
		&= 3 + 3x + 3x^2 + 2 - 4x^2 + 4x + 4x \\
		&= 5 + 5x - x^2 \\
		[p(x)]_{\beta} &= 5 + 5x - x^2 = c_1(1 + x - x^2) + c_2(x + x^2) + c_3(-x + 3x^2) \\
		[p(x)]_{\beta} &= \begin{cases}
			5 = c_1 + 0 + 0     \\
			5 = c_1 + c_2 - c_3 \\
			-1 = -c_1 + c_2 + 3c_3
		\end{cases}  \implies \begin{cases}
			c_1 = 5 \\
			c_2 = 1 \\
			c_3 = 1
		\end{cases} \\
		[p(x)]_{\beta} &= \begin{bmatrix} 5 \\ 1 \\ 1 \end{bmatrix}
	\end{split}
\]
\section{Answer of Given Questions}

In $R^2$, let $L$ be the line $y = mx$, where $m \neq 0$. Find an expression for $T(x, y)$, where
\begin{enumerate}
	\item $T$ is the reflection of $R^2$ about $L$.
	      \[
		      \begin{split}
			      M_x = \frac{x_1 + x_2}{2} \quad \text{and} \quad M_y = \frac{y_1 + y_2}{2} \\
			      AB = \frac{y_2 - y_1}{x_2 - x_1} = -1/m
		      \end{split}
	      \]
	\item $T$ is the projection on $L$ along the line perpe qqndicular to $L$. (See the definition of projection in the exercises of Section 2.1.)
\end{enumerate}

\section{Answer of Given Questions 2}

Let $V$ be a vector space and $S$ a subset of $V$ with the property that whenever $v_1, v_2, \cdots, v_n \in S$ and $a_1 v_1 + a_2 v_2 + \cdots + a_n v_n = 0$,
then $a_1 = a_2 = \cdots = a_n = 0$. Prove that every vector in the span of $S$ can be uniquely writen as a linear combination of vectors of $S$.

\section{Answer of Given Questions 3}
The combinattion $W= \{f(x) \in P(F): f(x) = 0 \text{ or } f(x) \text{ is a polynomial of degree n }\}$ is a subspace of $P(F)$ if $n \ge 1$. Justify your answer.

\section{Answer of Given Questions 4}
Let $T \in L(V, W)$. Since each subspace has a complement, we can write $V = N(T) \oplus N(T)^C$, where $N(T)^C$ is the complement of $N(T)$ in $V$. Show that any
complement of $N(T)$ is a isomorphic to $R(T)$.
\end{document}
