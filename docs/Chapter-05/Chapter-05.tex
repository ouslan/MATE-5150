\documentclass[10pt, oneside]{article}
\usepackage{amsmath, amsthm, amssymb, calrsfs, wasysym, verbatim, bbm, color, graphics, geometry}

\geometry{tmargin=.75in, bmargin=.75in, lmargin=.75in, rmargin = .75in}

\newcommand{\R}{\mathbb{R}}
\newcommand{\C}{\mathbb{C}}
\newcommand{\Z}{\mathbb{Z}}
\newcommand{\N}{\mathbb{N}}
\newcommand{\Q}{\mathbb{Q}}
\newcommand{\Cdot}{\boldsymbol{\cdot}}

\newtheorem{thm}{Theorem}
\newtheorem{defn}{Definition}
\newtheorem{conv}{Convention}
\newtheorem{rem}{Remark}
\newtheorem{lem}{Lemma}
\newtheorem{cor}{Corollary}


\title{MATE 5150: Determinants}
\author{Alejandro Ouslan}
\date{Academic Year 2024-2025}

\begin{document}

\maketitle
\tableofcontents

\vspace{.25in}

\section{Diagonalizability}

In Example 6 of Sectoin5.1, we obtained a basiss of eigenvectors by
choosing one eigenvector corresponding to each eigenvalue. In general,
such a procedure does not yeild a basis, but the following theorem shows 
that any set constructed in this manner is linearly independen. 

\begin{thm}
  Let $T$ be a linear operator on a vector space $V$, and let 
  $\lambda_1, \ldots, \lambda_m$ be distict eigencalues of $T$. If 
  $v_1, \ldots, v_m$, then $\{v_1, \ldots, v_m\}$ is linearly independent
\end{thm}

\begin{cor}
  Let $T$ be a linear operator on an n-dimensional vector space $V$.
  If $T$ has $n$ distict eigenalues, then $T$ is diagonalizable. 
  \end{cor}

\begin{proof}
  Supongamos que $T$ tiene $n$ valores propios distintos
  $\lambda_1, \ldots, \lambda_n$ y sea $v_1, \ldots, v_n$ vectores 
  propios de $T$ correspondientes a $\lambda_1, \ldots, \lambda_n$. para 
  $1 \leq i \leq n$. por el terorema 5.5 son linealmente independientes
  Dado que $dim(V)$ entonces el candidato para ser una base 
  de $V$ es $\{v_1, \ldots, v_n\}$. Por el terorema 5.1 entonce. $T$ es diagonalizable.
\end{proof}

\begin{defn}
  A polynomial $f(x)$ is $P(F)$ \textbf{splits over} $F$ if ther are 
  scalars $c, a_1, \ldots, a_n$ (not necessarlily distinct) in $F$ such that
  \begin{equation*}
    f(x) = c(t - a_1) \cdots (t - a_n)
  \end{equation*}
\end{defn}

\begin{thm}
  The characteristic polynomial of any diagonalizable linear operator splits.
  \end{thm}

\begin{proof}
  Supongamos $T$ es diagonalizable y $\exists$  una base $\beta$ 
  para $v$ tal que $[T]_{\beta} = D$, donde $D$ es diagonal y supongamos que 
  \[
    D = 
    \begin{bmatrix}
      \lambda_1 & 0 & \cdots & 0 \\
      0 & \lambda_2 & \cdots & 0 \\
      \vdots & \vdots & \ddots & \vdots \\
      0 & 0 & \cdots & \lambda_n \\
      \end{bmatrix}
  \]
  Sea $f(t)$ el polinomio caracteristico de $T$. Entonces 
  \begin{align*}
    f(t) &= det(D - tI_n) \\
    det \begin{bmatrix}
      \lambda_1 - t & 0 & \cdots & 0 \\
      0 & \lambda_2 - t & \cdots & 0 \\
      \vdots & \vdots & \ddots & \vdots \\
      0 & 0 & \cdots & \lambda_n - t \\
      \end{bmatrix} &= (\lambda_1 - t) \cdots (\lambda_n - t) \\
      \end{align*}
\end{proof}

\begin{defn}
  Let $\lambda$ be an eigenvalue of linear operator or matrix wiht characteristic 
  polynomial $f(t)$. The \textbf{algebraic multiplicity} of $\lambda$ is the largest
  positive integer $k$ for which $(t - \lambda)^k$ is a factor of $f(t)$.
\end{defn}

\begin{defn}
  

\end{document}
