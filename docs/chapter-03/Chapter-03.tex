\documentclass[10pt, oneside]{article}
\usepackage{amsmath, amsthm, amssymb, calrsfs, wasysym, verbatim, bbm, color, graphics, geometry}

\geometry{tmargin=.75in, bmargin=.75in, lmargin=.75in, rmargin = .75in}

\newcommand{\R}{\mathbb{R}}
\newcommand{\C}{\mathbb{C}}
\newcommand{\Z}{\mathbb{Z}}
\newcommand{\N}{\mathbb{N}}
\newcommand{\Q}{\mathbb{Q}}
\newcommand{\Cdot}{\boldsymbol{\cdot}}

\newtheorem{thm}{Theorem}
\newtheorem{defn}{Definition}
\newtheorem{conv}{Convention}
\newtheorem{rem}{Remark}
\newtheorem{lem}{Lemma}
\newtheorem{cor}{Corollary}


\title{MATE 5150: Elementary Matrix Operations and Systrems of Linear Equations}
\author{Alejandro Ouslan}
\date{Academic Year 2024-2025}

\begin{document}

\maketitle
\tableofcontents

\vspace{.25in}

\section{Elementary Matrix Operations and Elementary Matrices}

In this section, we difine the elementary operations that are used throughout the chapter. In subsequent sections, we use these
operations to obtrain simiple computational methods for determining the rank of a linear transformation and the solution of a
system of linear equations. There are two types of elementary operations -row operations and column operations. As we will see,
the row operations are more useful. They arise from the three operations that can be used to eliminate variables in a system of linear equations.

\subsection{Elementary Operations}
\begin{defn}
	Let $A$ be an $m \times n$ matrix. Any of the following three operations on the rows [columns] of $A$ is called an \textbf{elementary row [column] operation}:
	\begin{enumerate}
		\item interchanging any two rows [columns] of $A$;
		\item multiplying any row [column] of $A$ by a nonzero scalar;
		\item adding any scalar multiple of a row [column] of $A$ to another row [column] of $A$.
	\end{enumerate}
\end{defn}

Any of these three operations is called an \textbf{elementary operation}. Elementary operations are of \textbf{type 1}, \textbf{type 2}, or \textbf{type 3} depending on
wheather they are obtained by (1), (2), or (3) above.

\begin{defn}
	An $n \times n$ \textbf{elementary matrix} is a matrix obtained by performing an elementary operations on $I_n$. The elemetnary matrix is said to be of \textbf{type 1}, \textbf{type 2}, or \textbf{type 3}
	acording to wheather the elementary operation performed on $I_n$ is of type 1, type 2, or type 3.
\end{defn}

\subsubsection{Example Elementary Matrices}
Let
\[
	A = \begin{bmatrix} 1 & 2 & 3 \\ 1 & 1 & 1 \\ 1 & -1 & 1 \end{bmatrix} \quad ,
	B = \begin{bmatrix} 1 & 0 & 3 \\ 1 & -2 & 1 \\ 1 & -3 & 1 \end{bmatrix} \quad \text{and} \quad ,
	C = \begin{bmatrix} 1 & 0 & 0 \\ 0 & -2 & -2 \\ 1 & -3 & 1 \end{bmatrix}
\]

Find an elementary operation that transforms $A$ into $B$ and an elementary operation that transforms $B$ into $C$. By means of several additional operations, transform C into $I_3$.
\[
	\begin{split}
		AE = B \quad & \text{where} \quad E = \begin{bmatrix} 1 & -2 & 0 \\ 0 & 1 & 0 \\ 0 & 0 & 1 \end{bmatrix} \\
		EB = C \quad & \text{where} \quad E = \begin{bmatrix} 1 & 0 & 0 \\ -1 & 1 & 0 \\ 0 & 0 & 1 \end{bmatrix} \\
		E_1E_2E_3E_4 = I_3 \quad & \text{where} \quad E_1 = \begin{bmatrix} 1 & 0 & 0 \\ 0 & 1 & 0 \\ -1 & 0 & 1 \end{bmatrix} \quad ,
		\quad E_2 = \begin{bmatrix} 1 & 0 & 0 \\ 0 & -\frac{1}{2} & 0 \\ 0 & 0 & 1 \end{bmatrix} \quad ,
		\quad E_3 = \begin{bmatrix} 1 & 0 & 0 \\ 0 & 1 & 0 \\ 0 & 3 & 1 \end{bmatrix} \quad , E_4 = \begin{bmatrix} 1 & 0 & 3 \\ 0 & 1 & 1 \\ 0 & 0 & 1 \end{bmatrix}
	\end{split}
\]

\subsection{Properties of Elementary Matrices}
\begin{thm}
	Elementary matrices are invertible, and the inverse of an elementary matrix is an elementary matrix of the same type.
\end{thm}

\begin{proof}
	\begin{align*}
		\intertext{Let $E$ be an elementary matrix $n \times n$. The $E$ is defided by an elementary operation on $I_n$.}
	\end{align*}
\end{proof}

\subsubsection{Example}
Use the proog in Theroem 1 to obtain the inverse of each of the folowing elementary matrices:
\[
	A = \begin{bmatrix} 0 & 0 & 1 \\ 0 & 1 & 0 \\ 0 & 0 & 1 \end{bmatrix}, \quad
	B = \begin{bmatrix} 1 & 0 & 0 \\ 0 & 3 & 0 \\ 0 & 0 & 1 \end{bmatrix}, \quad
	C = \begin{bmatrix} 1 & 0 & 0 \\ 0 & 1 & 0 \\ -2 & 0 & 1 \end{bmatrix}
\]
Finding the inverse of each of the elementary matrices in the example above, we have:
\[
	\begin{split}
		E &= \begin{bmatrix} 0 & 0 & 1 \\ 0 & 1 & 0 \\ 1 & 0 & 0 \end{bmatrix} \quad , \text{Therefore} \quad A^{-1} = \begin{bmatrix} 0 & 0 & 1 \\ 0 & 1 & 0 \\ 1 & 0 & 0 \end{bmatrix} \\
		E &= \begin{bmatrix} 1 & 0 & 0 \\ 0 & \frac{1}{3} & 0 \\ 0 & 0 & 1 \end{bmatrix} \quad , \text{Therefore} \quad B^{-1} = \begin{bmatrix} 1 & 0 & 0 \\ 0 & \frac{1}{3} & 0 \\ 0 & 0 & 1 \end{bmatrix} \\
		E &= \begin{bmatrix} 1 & 0 & 0 \\ 0 & 1 & 0 \\ 2 & 0 & 1 \end{bmatrix} \quad , \text{Therefore} \quad C^{-1} = \begin{bmatrix} 1 & 0 & 0 \\ 0 & 1 & 0 \\ -2 & 0 & 1 \end{bmatrix}
	\end{split}
\]

\subsection{Matrix Multiplication and Elementary Matrices}

Let $A$ be an $m \times n$ matrix. Prove that if $E$ can be otained from $A$ by an elementary row [column] operation, then $B^T$ can be obtained from $A^T$ by the corresponding elementary column [row] operation.

\begin{proof}
	\begin{align*}
		(E_RB)^T & = (A)^T \\
		B^TE_R^T & = A^T
	\end{align*}
	Therefore, $B^T$ can be obtained from $A^T$ by the corresponding elementary column operation.
\end{proof}


\end{document}
