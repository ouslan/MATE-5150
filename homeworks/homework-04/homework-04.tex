
\documentclass{article}

\usepackage{fancyhdr}
\usepackage{extramarks}
\usepackage{amsmath}
\usepackage{amsthm}
\usepackage{amsfonts}
\usepackage{tikz}
\usepackage[plain]{algorithm}
\usepackage{algpseudocode}

\usetikzlibrary{automata,positioning}

%
% Basic Document Settings
%

\topmargin=-0.45in
\evensidemargin=0in
\oddsidemargin=0in
\textwidth=6.5in
\textheight=9.0in
\headsep=0.25in

\linespread{1.1}

\pagestyle{fancy}
\lhead{\hmwkAuthorName}
\chead{\hmwkClass\ (\hmwkClassInstructor): \hmwkTitle}
\rhead{\firstxmark}
\lfoot{\lastxmark}
\cfoot{\thepage}

\renewcommand\headrulewidth{0.4pt}
\renewcommand\footrulewidth{0.4pt}

\setlength\parindent{0pt}

%
% Create Problem Sections
%

\newcommand{\enterProblemHeader}[1]{
	\nobreak\extramarks{}{Problem \arabic{#1} continued on next page\ldots}\nobreak{}
	\nobreak\extramarks{Problem \arabic{#1} (continued)}{Problem \arabic{#1} continued on next page\ldots}\nobreak{}
}

\newcommand{\exitProblemHeader}[1]{
	\nobreak\extramarks{Problem \arabic{#1} (continued)}{Problem \arabic{#1} continued on next page\ldots}\nobreak{}
	\stepcounter{#1}
	\nobreak\extramarks{Problem \arabic{#1}}{}\nobreak{}
}

\setcounter{secnumdepth}{0}
\newcounter{partCounter}
\newcounter{homeworkProblemCounter}
\setcounter{homeworkProblemCounter}{1}
\nobreak\extramarks{Problem \arabic{homeworkProblemCounter}}{}\nobreak{}

%
% Homework Problem Environment
%
% This environment takes an optional argument. When given, it will adjust the
% problem counter. This is useful for when the problems given for your
% assignment aren't sequential. See the last 3 problems of this template for an
% example.
%
\newenvironment{homeworkProblem}[1][-1]{
	\ifnum#1>0
		\setcounter{homeworkProblemCounter}{#1}
	\fi
	\section{Problem \arabic{homeworkProblemCounter}}
	\setcounter{partCounter}{1}
	\enterProblemHeader{homeworkProblemCounter}
}{
	\exitProblemHeader{homeworkProblemCounter}
}

%
% Homework Details
%   - Title
%   - Due date
%   - Class
%   - Section/Time
%   - Instructor
%   - Author
%

\newcommand{\hmwkTitle}{Asignacion\ \#3}
\newcommand{\hmwkDueDate}{Septiembre 26, 2024}
\newcommand{\hmwkClass}{MATE 5150}
\newcommand{\hmwkClassInstructor}{Dr. Pedro Vasquez}
\newcommand{\hmwkAuthorName}{\textbf{Alejandro Ouslan}}

%
% Title Page
%

\title{
	\vspace{2in}
	\textmd{\textbf{\hmwkClass:\ \hmwkTitle}}\\
	\normalsize\vspace{0.1in}\small{Due\ on\ \hmwkDueDate}\\
	\vspace{0.1in}\large{\textit{\hmwkClassInstructor}}
	\vspace{3in}
}

\author{\hmwkAuthorName}
\date{}

\renewcommand{\part}[1]{\textbf{\large Part \Alph{partCounter}}\stepcounter{partCounter}\\}

%
% Various Helper Commands
%

% Useful for algorithms
\newcommand{\alg}[1]{\textsc{\bfseries \footnotesize #1}}

% For derivatives
\newcommand{\deriv}[1]{\frac{\mathrm{d}}{\mathrm{d}x} (#1)}

% For partial derivatives
\newcommand{\pderiv}[2]{\frac{\partial}{\partial #1} (#2)}

% Integral dx
\newcommand{\dx}{\mathrm{d}x}

% Alias for the Solution section header
\newcommand{\solution}{\textbf{\large Solution}}

% Probability commands: Expectation, Variance, Covariance, Bias
\newcommand{\E}{\mathrm{E}}
\newcommand{\Var}{\mathrm{Var}}
\newcommand{\Cov}{\mathrm{Cov}}
\newcommand{\Bias}{\mathrm{Bias}}

\begin{document}

\maketitle

\pagebreak

% Homework problem 1 sec 2.4, prob 2b
\begin{homeworkProblem}
	Determine whether $T$ is invertible and justify your answer.
	$$T: \mathbb{R}^2 \rightarrow \mathbb{R}^3, T(a_1, a_2) = (3a_1 - a_2, a_2,4a_1)$$
\end{homeworkProblem}

% Homework problem 2 sec 2.4, prob 2f
\begin{homeworkProblem}
	Determine whether $T$ is invertible and justify your answer.
	\[
		T: M_{2 \times 2}(\mathbb{R}) \rightarrow M_{2 \times 2}(\mathbb{R}),
		T(\begin{bmatrix} a & b \\ c & d \end{bmatrix}) = \begin{bmatrix} a + b & a \\ c & c + d \end{bmatrix}
	\]
\end{homeworkProblem}

% Homework problem 3 sec 2.4, prob 3d
\begin{homeworkProblem}
	Is the following parirs of vector spaces are isomorphic? Justify your answer.
	$$V = \{A \in M_{2 \times 2}(\mathbb{R}): tr(A) = 0\} \text{ and } R^4$$
\end{homeworkProblem}

% Homework problem 4 sec 2.4, prob 14
\begin{homeworkProblem}
	Let
	\[
		V = \left\{ \begin{bmatrix} a & a+b \\ 0 & c \end{bmatrix} : a, b, c \in F \right\}
	\]
	Construct an isomorphism from $V$ to $F^3$.
\end{homeworkProblem}

% Homework problem 5 sec 2.5, prob 2b
\begin{homeworkProblem}
	For each of the following pairs of ordered bases $\beta$ and $\beta'$ for $\mathbb{R}^2$,
	find the change of coordinates matrix that changes $\beta$-coordinates into $\beta$-coordinate
	$$ \beta = \{(-1,3), (2,-1)\} \text{ and } \beta' = \{(0,10),(5,0)\}$$
\end{homeworkProblem}

% Homework problem 6 sec 2.5, prob 3b
\begin{homeworkProblem}
	For each of the folloing pairs of ordered bases $\beta$ and $\beta'$ for $P_2(\mathbb{R})$,
	find the change of coorinate matrix that changs $\beta'$-coordinates into $\beta$-coordinates.
	$$\beta = \{1,x,x^2\} \text{ and } \beta' = \{a_2x^2 + a_1x + a_0, b_2x^2 + b_1x + b_0, c_2x^2 + c_1x + c_0\}$$
\end{homeworkProblem}

% Homework problem 7 sec 2.5, prob 4
\begin{homeworkProblem}
	Let $T$ be the linear operator on $\mathbb{R}^2$ defined by
	$$T\begin{pmatrix} a \\ b \end{pmatrix} = \begin{pmatrix} 2a + b \\ a - 3b \end{pmatrix}$$
	let $\beta$ be the standard ordered basis for $\mathbb{R}^2$, and let
	$$\beta' = \left\{ \begin{pmatrix} 1 \\ 1 \end{pmatrix}, \begin{pmatrix} 1 \\ 2 \end{pmatrix} \right\}$$
	Use Theorem 2.23 and the fact that
	$$\begin{pmatrix} 1 & 1 \\ 1 & 2 \end{pmatrix}^{-1} = \begin{pmatrix} 2 & -1 \\ -1 & 1 \end{pmatrix}$$
	To find $[T]_{\beta'}$.
\end{homeworkProblem}

% Homework problem 8 sec 2.5, prob 6b
\begin{homeworkProblem}
	For each matrix $A$ and ordered basis $\beta$, find $[L_A]_{\beta}$. Also, find and
	invertible matrix $Q$ such that $[L_A]_{\beta} = Q^{-1}AQ$.
	$$A = \begin{pmatrix} 1 & 2 \\ 2 & 1 \end{pmatrix} \text{ and } \beta = \left\{ \begin{pmatrix} 1 \\ 1 \end{pmatrix}, \begin{pmatrix} 1 \\ -1 \end{pmatrix} \right\}$$
\end{homeworkProblem}

\end{document}
