\documentclass{article}

\usepackage{fancyhdr}
\usepackage{extramarks}
\usepackage{amsmath}
\usepackage{amsthm}
\usepackage{amsfonts}
\usepackage{tikz}
\usepackage[plain]{algorithm}
\usepackage{algpseudocode}

\usetikzlibrary{automata,positioning}

%
% Basic Document Settings
%

\topmargin=-0.45in
\evensidemargin=0in
\oddsidemargin=0in
\textwidth=6.5in
\textheight=9.0in
\headsep=0.25in

\linespread{1.1}

\pagestyle{fancy}
\lhead{\hmwkAuthorName}
\chead{\hmwkClass\ (\hmwkClassInstructor): \hmwkTitle}
\rhead{\firstxmark}
\lfoot{\lastxmark}
\cfoot{\thepage}

\renewcommand\headrulewidth{0.4pt}
\renewcommand\footrulewidth{0.4pt}

\setlength\parindent{0pt}

%
% Create Problem Sections
%

\newcommand{\enterProblemHeader}[1]{
	\nobreak\extramarks{}{Problem \arabic{#1} continued on next page\ldots}\nobreak{}
	\nobreak\extramarks{Problem \arabic{#1} (continued)}{Problem \arabic{#1} continued on next page\ldots}\nobreak{}
}

\newcommand{\exitProblemHeader}[1]{
	\nobreak\extramarks{Problem \arabic{#1} (continued)}{Problem \arabic{#1} continued on next page\ldots}\nobreak{}
	\stepcounter{#1}
	\nobreak\extramarks{Problem \arabic{#1}}{}\nobreak{}
}

\setcounter{secnumdepth}{0}
\newcounter{partCounter}
\newcounter{homeworkProblemCounter}
\setcounter{homeworkProblemCounter}{1}
\nobreak\extramarks{Problem \arabic{homeworkProblemCounter}}{}\nobreak{}

%
% Homework Problem Environment
%
% This environment takes an optional argument. When given, it will adjust the
% problem counter. This is useful for when the problems given for your
% assignment aren't sequential. See the last 3 problems of this template for an
% example.
%
\newenvironment{homeworkProblem}[1][-1]{
	\ifnum#1>0
		\setcounter{homeworkProblemCounter}{#1}
	\fi
	\section{Problem \arabic{homeworkProblemCounter}}
	\setcounter{partCounter}{1}
	\enterProblemHeader{homeworkProblemCounter}
}{
	\exitProblemHeader{homeworkProblemCounter}
}

%
% Homework Details
%   - Title
%   - Due date
%   - Class
%   - Section/Time
%   - Instructor
%   - Author
%

\newcommand{\hmwkTitle}{Asignacion\ \#3}
\newcommand{\hmwkDueDate}{Septiembre 26, 2024}
\newcommand{\hmwkClass}{MATE 5150}
\newcommand{\hmwkClassInstructor}{Dr. Pedro Vasquez}
\newcommand{\hmwkAuthorName}{\textbf{Alejandro Ouslan}}

%
% Title Page
%

\title{
	\vspace{2in}
	\textmd{\textbf{\hmwkClass:\ \hmwkTitle}}\\
	\normalsize\vspace{0.1in}\small{Due\ on\ \hmwkDueDate}\\
	\vspace{0.1in}\large{\textit{\hmwkClassInstructor}}
	\vspace{3in}
}

\author{\hmwkAuthorName}
\date{}

\renewcommand{\part}[1]{\textbf{\large Part \Alph{partCounter}}\stepcounter{partCounter}\\}

%
% Various Helper Commands
%

% Useful for algorithms
\newcommand{\alg}[1]{\textsc{\bfseries \footnotesize #1}}

% For derivatives
\newcommand{\deriv}[1]{\frac{\mathrm{d}}{\mathrm{d}x} (#1)}

% For partial derivatives
\newcommand{\pderiv}[2]{\frac{\partial}{\partial #1} (#2)}

% Integral dx
\newcommand{\dx}{\mathrm{d}x}

% Alias for the Solution section header
\newcommand{\solution}{\textbf{\large Solution}}

% Probability commands: Expectation, Variance, Covariance, Bias
\newcommand{\E}{\mathrm{E}}
\newcommand{\Var}{\mathrm{Var}}
\newcommand{\Cov}{\mathrm{Cov}}
\newcommand{\Bias}{\mathrm{Bias}}

\begin{document}

\maketitle

\pagebreak

% Homework problem 1 sec 2.1, prob 3
\begin{homeworkProblem}
	Prove that $T$ is a linear transformation, and find bases for both $N(T)$ and $R(T)$. Then compute the
	nulity and rank of $T$, and verify the dimension theorem. Finally use the appropriate theorems in this section
	to determine wheather $T$ is one-to-one or onto.
	$$T:R^2 \rightarrow R^3 \text{ defined by } T(a_1, a_2) = (a_1 + 2a_2, 0, 2a_1 - a_2)$$
\end{homeworkProblem}

% Homework problem 2 sec 2.1, prob 4
\begin{homeworkProblem}
	Prove that $T$ is a linear transformation, and find bases for both $N(T)$ and $R(T)$. Then compute the
	nulity and rank of $T$, and verify the dimension theorem. Finally use the appropriate theorems in this section
	to determine wheather $T$ is one-to-one or onto.
	\[
		T:M_{2x3} (F) \rightarrow M_{2x2} (F) \text{ defined by } T
		\begin{pmatrix}
			a_{11} & a_{12} & a_{13} \\
			a_{21} & a_{22} & a_{23}
		\end{pmatrix} =
		\begin{pmatrix}
			2a_{11} - a_{12} & a_{13} + 2a_{12} \\
			0                & 0
		\end{pmatrix}
	\]
\end{homeworkProblem}

% Homework problem 3 sec 2.1, prob 9e
\begin{homeworkProblem}
	In this exercise, $T:R^2 \rightarrow R^2$ is a function. State why $T$ is not linear.
	$$T(a_1, a_2) = (a_1 + 1, a_2)$$
\end{homeworkProblem}

% Homework problem 4 sec 2.1, prob 12
\begin{homeworkProblem}
	Is there a linear transformation $T:R^3 \rightarrow R^2$ such that $T(1, 0, 3) = (1,1)$ and $T(-2,0, -6) = (2,1)$?
\end{homeworkProblem}

% Homework problem 5 sec 2.1, prob 13
\begin{homeworkProblem}
	Let $V$ and $W$ be vector spaces, let $T:V \rightarrow W$ be a linear, and let $\{w_1, w_2,\ldots,w_k\}$
	be a linearly independent set of $k$ vectors from $R(T)$. Prove that if $S = \{v_1, v_2,\ldots,v_k\}$ is chosen
	so that $T(v_i) = w_i$ for $i = 1,2,\ldots,k$, then $S$ is linearly independent. Visit goo.gl/kmaQS2 for a solution.
\end{homeworkProblem}

% Homework problem 6 sec 2.2, prob 2
\begin{homeworkProblem}
	Let $\beta$ and $\gamma$ be the standard ordered bases for $R^n$ and $R^m$, respectively. For each linear
	transformation $T:R^n \rightarrow R^m$, compute $[T]^{\gamma}_{\beta}$.
	\begin{enumerate}
		\item $T:R^3 \rightarrow R \text{ defined by } T(a_1, a_2, a_3) = 2a_1 + a_2 - 3a_3 $
		\item $T:R^n \rightarrow R^n \text{ defined by } T(a_1, a_2, \ldots, a_n) = (a_n, a_{n-1},\ldots,a_1)$
	\end{enumerate}
\end{homeworkProblem}

\begin{homeworkProblem}
\end{homeworkProblem}
\end{document}


