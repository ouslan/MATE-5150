\documentclass{article}

\usepackage{fancyhdr}
\usepackage{extramarks}
\usepackage{amsmath}
\usepackage{amsthm}
\usepackage{amsfonts}
\usepackage{tikz}
\usepackage[plain]{algorithm}
\usepackage{algpseudocode}

\usetikzlibrary{automata,positioning}

%
% Basic Document Settings
%

\topmargin=-0.45in
\evensidemargin=0in
\oddsidemargin=0in
\textwidth=6.5in
\textheight=9.0in
\headsep=0.25in

\linespread{1.1}

\pagestyle{fancy}
\lhead{\hmwkAuthorName}
\chead{\hmwkClass\ (\hmwkClassInstructor): \hmwkTitle}
\rhead{\firstxmark}
\lfoot{\lastxmark}
\cfoot{\thepage}

\renewcommand\headrulewidth{0.4pt}
\renewcommand\footrulewidth{0.4pt}

\setlength\parindent{0pt}

%
% Create Problem Sections
%

\newcommand{\enterProblemHeader}[1]{
	\nobreak\extramarks{}{Problem \arabic{#1} continued on next page\ldots}\nobreak{}
	\nobreak\extramarks{Problem \arabic{#1} (continued)}{Problem \arabic{#1} continued on next page\ldots}\nobreak{}
}

\newcommand{\exitProblemHeader}[1]{
	\nobreak\extramarks{Problem \arabic{#1} (continued)}{Problem \arabic{#1} continued on next page\ldots}\nobreak{}
	\stepcounter{#1}
	\nobreak\extramarks{Problem \arabic{#1}}{}\nobreak{}
}

\setcounter{secnumdepth}{0}
\newcounter{partCounter}
\newcounter{homeworkProblemCounter}
\setcounter{homeworkProblemCounter}{1}
\nobreak\extramarks{Problem \arabic{homeworkProblemCounter}}{}\nobreak{}

%
% Homework Problem Environment
%
% This environment takes an optional argument. When given, it will adjust the
% problem counter. This is useful for when the problems given for your
% assignment aren't sequential. See the last 3 problems of this template for an
% example.
%
\newenvironment{homeworkProblem}[1][-1]{
	\ifnum#1>0
		\setcounter{homeworkProblemCounter}{#1}
	\fi
	\section{Problem \arabic{homeworkProblemCounter}}
	\setcounter{partCounter}{1}
	\enterProblemHeader{homeworkProblemCounter}
}{
	\exitProblemHeader{homeworkProblemCounter}
}

%
% Homework Details
%   - Title
%   - Due date
%   - Class
%   - Section/Time
%   - Instructor
%   - Author
%

\newcommand{\hmwkTitle}{Asignacion\ \#7}
\newcommand{\hmwkDueDate}{Noviembre 14, 2024}
\newcommand{\hmwkClass}{MATE 5150}
\newcommand{\hmwkClassInstructor}{Dr. Pedro Vasquez}
\newcommand{\hmwkAuthorName}{\textbf{Alejandro Ouslan}}

%
% Title Page
%

\title{
	\vspace{2in}
	\textmd{\textbf{\hmwkClass:\ \hmwkTitle}}\\
	\normalsize\vspace{0.1in}\small{Due\ on\ \hmwkDueDate}\\
	\vspace{0.1in}\large{\textit{\hmwkClassInstructor}}
	\vspace{3in}
}

\author{\hmwkAuthorName}
\date{}

\renewcommand{\part}[1]{\textbf{\large Part \Alph{partCounter}}\stepcounter{partCounter}\\}

%
% Various Helper Commands
%

% Useful for algorithms
\newcommand{\alg}[1]{\textsc{\bfseries \footnotesize #1}}

% For derivatives
\newcommand{\deriv}[1]{\frac{\mathrm{d}}{\mathrm{d}x} (#1)}

% For partial derivatives
\newcommand{\pderiv}[2]{\frac{\partial}{\partial #1} (#2)}

% Integral dx
\newcommand{\dx}{\mathrm{d}x}

% Alias for the Solution section header
\newcommand{\solution}{\textbf{\large Solution}}

% Probability commands: Expectation, Variance, Covariance, Bias
\newcommand{\E}{\mathrm{E}}
\newcommand{\Var}{\mathrm{Var}}
\newcommand{\Cov}{\mathrm{Cov}}
\newcommand{\Bias}{\mathrm{Bias}}

\begin{document}

\maketitle

\pagebreak

% Homework problem 1 sec 4.1 problem 3b
\begin{homeworkProblem}
	Compute the determinants of the following matrices in $M_{2x2}(C)$.
	\[
		\begin{bmatrix}
			2i & 3  \\
			4  & 6i
		\end{bmatrix}
	\]
\end{homeworkProblem}

% Homework problem 2 sec 4.1 problem 12
\begin{homeworkProblem}
	The classical adjoint of a $2x2$ matrix $A \in M_{2x2}(F)$ is the matrix
	\[
		C = \begin{bmatrix}
			A_{22}  & -A_{12} \\
			-A_{21} & A_{11}
		\end{bmatrix}
	\]
	Prove that
	\begin{itemize}
		\item $C A = A C = [det(A)]I$
		\item $det(C) = det(A)$
		\item The classical adjoint of $A^T$ is $C^T$
		\item If $A$ is invertible, then $A^{-1} = [det(A)]^{-1}C$
	\end{itemize}
\end{homeworkProblem}

% Homework problem 3 sec 4.2 problem 4
\begin{homeworkProblem}
	\[
		det \begin{bmatrix}
			b_1 + c_1 & b_2 + c_2 & b_3 + c_3 \\
			a_1 + c_1 & a_2 + c_2 & a_3 + c_3 \\
			a_1 + b_1 & a_2 + b_2 & a_3 + b_3 \\
		\end{bmatrix} = k \cdot det \begin{bmatrix}
			a_1 & a_2 & a_3 \\
			b_1 & b_2 & b_3 \\
			c_1 & c_2 & c_3 \\
		\end{bmatrix}
	\]
\end{homeworkProblem}

% Homework problem 4 sec 4.2 problem 12
\begin{homeworkProblem}
	Evaluate the determinant of the given matrix by cofactor expansion along the indicated row.
	\[
		\begin{bmatrix}
			1  & -1 & 2  & -1 \\
			-3 & 4  & 1  & -1 \\
			2  & -5 & -3 & 8  \\
			-2 & 6  & -4 & 1  \\
		\end{bmatrix}
	\]
	along the fourth row.
\end{homeworkProblem}

% Homework problem 5 sec 4.2 problem 28
\begin{homeworkProblem}
	Compute $det(E_i)$ if $E_i$ is an elementary matrix of type $i$.
\end{homeworkProblem}

% Homework problem 6 sec 4.3 problem 5
\begin{homeworkProblem}
	Use Cramer's rule to solve the given system of linear equations.
	\[
		\begin{aligned}
			x_1 - x_2 + 4x_3   & = -4 \\
			-8x_1 + 3x_2 + x_3 & = 8  \\
			2x_1 + x_2 + x_3   & = 0
		\end{aligned}
	\]
\end{homeworkProblem}

% Homework problem 7 sec 4.3 problem 10
\begin{homeworkProblem}
	A matrix $M \in M_{nxn}(F)$ is called nipotent if, some positive interrger $k$, $M^k = 0$, where $0$ is the $n x n$ zero matrix. Prove that if $M$ is nilpotent, then $det(M) = 0$.
\end{homeworkProblem}

% Homework problem 8 sec 4.3 problem 16
\begin{homeworkProblem}
	Use determinants to prove that if $A,B \in M_{nxn}(F)$ are sucht that $AB = I$, then $A$ is invertible (and hence $B = A^{-1}$).
\end{homeworkProblem}

% Homework problem 9 sec 4.3 problem 24
\begin{homeworkProblem}
	Let $A \in M_{nxn}(F)$ have the form
	\[
		A = \begin{bmatrix}
			0      & 0      & 0      & \cdots & 0      & a_{0}   \\
			-1     & 0      & 0      & \cdots & 0      & a_{1}   \\
			0      & -1     & 0      & \cdots & 0      & a_{2}   \\
			\vdots & \vdots & \vdots & \ddots & \vdots & \vdots  \\
			0      & 0      & 0      & \cdots & -1     & a_{n-1}
		\end{bmatrix}
	\]

	Compute $det(A + tI)$ where $I$ is the $n x n$ identity matrix.
\end{homeworkProblem}

% Homework problem 10 sec 4.4 Section 4.4 problem 4th
\begin{homeworkProblem}
\end{homeworkProblem}



\end{document}
