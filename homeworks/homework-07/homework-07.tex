\documentclass{article}

\usepackage{fancyhdr}
\usepackage{extramarks}
\usepackage{amsmath}
\usepackage{amsthm}
\usepackage{amsfonts}
\usepackage{tikz}
\usepackage[plain]{algorithm}
\usepackage{algpseudocode}

\usetikzlibrary{automata,positioning}

%
% Basic Document Settings
%

\topmargin=-0.45in
\evensidemargin=0in
\oddsidemargin=0in
\textwidth=6.5in
\textheight=9.0in
\headsep=0.25in

\linespread{1.1}

\pagestyle{fancy}
\lhead{\hmwkAuthorName}
\chead{\hmwkClass\ (\hmwkClassInstructor): \hmwkTitle}
\rhead{\firstxmark}
\lfoot{\lastxmark}
\cfoot{\thepage}

\renewcommand\headrulewidth{0.4pt}
\renewcommand\footrulewidth{0.4pt}

\setlength\parindent{0pt}

%
% Create Problem Sections
%

\newcommand{\enterProblemHeader}[1]{
	\nobreak\extramarks{}{Problem \arabic{#1} continued on next page\ldots}\nobreak{}
	\nobreak\extramarks{Problem \arabic{#1} (continued)}{Problem \arabic{#1} continued on next page\ldots}\nobreak{}
}

\newcommand{\exitProblemHeader}[1]{
	\nobreak\extramarks{Problem \arabic{#1} (continued)}{Problem \arabic{#1} continued on next page\ldots}\nobreak{}
	\stepcounter{#1}
	\nobreak\extramarks{Problem \arabic{#1}}{}\nobreak{}
}

\setcounter{secnumdepth}{0}
\newcounter{partCounter}
\newcounter{homeworkProblemCounter}
\setcounter{homeworkProblemCounter}{1}
\nobreak\extramarks{Problem \arabic{homeworkProblemCounter}}{}\nobreak{}

%
% Homework Problem Environment
%
% This environment takes an optional argument. When given, it will adjust the
% problem counter. This is useful for when the problems given for your
% assignment aren't sequential. See the last 3 problems of this template for an
% example.
%
\newenvironment{homeworkProblem}[1][-1]{
	\ifnum#1>0
		\setcounter{homeworkProblemCounter}{#1}
	\fi
	\section{Problem \arabic{homeworkProblemCounter}}
	\setcounter{partCounter}{1}
	\enterProblemHeader{homeworkProblemCounter}
}{
	\exitProblemHeader{homeworkProblemCounter}
}

%
% Homework Details
%   - Title
%   - Due date
%   - Class
%   - Section/Time
%   - Instructor
%   - Author
%

\newcommand{\hmwkTitle}{Asignacion\ \#7}
\newcommand{\hmwkDueDate}{Noviembre 14, 2024}
\newcommand{\hmwkClass}{MATE 5150}
\newcommand{\hmwkClassInstructor}{Dr. Pedro Vasquez}
\newcommand{\hmwkAuthorName}{\textbf{Alejandro Ouslan}}

%
% Title Page
%

\title{
	\vspace{2in}
	\textmd{\textbf{\hmwkClass:\ \hmwkTitle}}\\
	\normalsize\vspace{0.1in}\small{Due\ on\ \hmwkDueDate}\\
	\vspace{0.1in}\large{\textit{\hmwkClassInstructor}}
	\vspace{3in}
}

\author{\hmwkAuthorName}
\date{}

\renewcommand{\part}[1]{\textbf{\large Part \Alph{partCounter}}\stepcounter{partCounter}\\}

%
% Various Helper Commands
%

% Useful for algorithms
\newcommand{\alg}[1]{\textsc{\bfseries \footnotesize #1}}

% For derivatives
\newcommand{\deriv}[1]{\frac{\mathrm{d}}{\mathrm{d}x} (#1)}

% For partial derivatives
\newcommand{\pderiv}[2]{\frac{\partial}{\partial #1} (#2)}

% Integral dx
\newcommand{\dx}{\mathrm{d}x}

% Alias for the Solution section header
\newcommand{\solution}{\textbf{\large Solution}}

% Probability commands: Expectation, Variance, Covariance, Bias
\newcommand{\E}{\mathrm{E}}
\newcommand{\Var}{\mathrm{Var}}
\newcommand{\Cov}{\mathrm{Cov}}
\newcommand{\Bias}{\mathrm{Bias}}

\begin{document}

\maketitle

\pagebreak

% Homework problem 1 sec 4.1 problem 3b
\begin{homeworkProblem}
	Compute the determinants of the following matrices in $M_{2x2}(C)$.
	\[
		\begin{bmatrix}
			2i & 3  \\
			4  & 6i
		\end{bmatrix}
	\]
	\[
		\begin{split}
			|C| & = 2i \cdot 6i - 3 \cdot 4 \\
			|C| & = 12i^2 - 12              \\
			|C| & = -12 - 12                \\
			|C| & = -24
		\end{split}
	\]
\end{homeworkProblem}

% Homework problem 2 sec 4.1 problem 12
\begin{homeworkProblem}
	The classical adjoint of a $2x2$ matrix $A \in M_{2x2}(F)$ is the matrix
	\[
		C = \begin{bmatrix}
			A_{22}  & -A_{12} \\
			-A_{21} & A_{11}
		\end{bmatrix}
	\]
	Prove that
	\begin{itemize}
		\item $C A = A C = [det(A)]I$
		      \[
			      \begin{split}
				      CA        & = \begin{bmatrix}
					                    A_{22}  & -A_{12} \\
					                    -A_{21} & A_{11}
				                    \end{bmatrix}
				      \begin{bmatrix}
					      A_{11} & A_{12} \\
					      A_{21} & A_{22}
				      \end{bmatrix} = \begin{bmatrix}
					                      A_{22}A_{11} - A_{12}A_{21} & 0                            \\
					                      0                           & -A_{21}A_{12} + A_{11}A_{22}
				                      \end{bmatrix}       \\
				      AC        & = \begin{bmatrix}
					                    A_{11} & A_{12} \\
					                    A_{21} & A_{22}
				                    \end{bmatrix}
				      \begin{bmatrix}
					      A_{22}  & -A_{12} \\
					      -A_{21} & A_{11}
				      \end{bmatrix} = \begin{bmatrix}
					                      A_{11}A_{22} - A_{12}A_{21} & 0                            \\
					                      0                           & -A_{21}A_{12} + A_{11}A_{22}
				                      \end{bmatrix}       \\
				      \det(A)   & = (A_{22}A_{11} - A_{12}A_{21}) \cdot (-A_{21}A_{12} + A_{11}A_{22}) \\
				      \det(A) I & = \begin{bmatrix}
					                    A_{22}A_{11} - A_{12}A_{21} & 0                            \\
					                    0                           & -A_{21}A_{12} + A_{11}A_{22}
				                    \end{bmatrix}
			      \end{split}
		      \]
		\item $det(C) = det(A)$
		      \[
			      \begin{split}
				      det(C) & = A_{22}A_{11} - A_{12}A_{21} \\
				      det(A) & = A_{11}A_{22} - A_{12}A_{21}
			      \end{split}
		      \]
		\item The classical adjoint of $A^T$ is $C^T$
		      \[
			      \begin{split}
				      A^T & = \begin{bmatrix}
					              A_{11} & A_{21} \\
					              A_{12} & A_{22}
				              \end{bmatrix}   \\
				      C^T & = \begin{bmatrix}
					              A_{22}  & -A_{21} \\
					              -A_{12} & A_{11}
				              \end{bmatrix}
			      \end{split}
		      \]
		\item If $A$ is invertible, then $A^{-1} = [det(A)]^{-1}C$
		      \[
			      \begin{split}
				      A^{-1} & = \frac{1}{det(A)} \begin{bmatrix}
					                                  A_{22}  & -A_{12} \\
					                                  -A_{21} & A_{11}
				                                  \end{bmatrix}                                      \\
				      A^{-1} & = \begin{bmatrix}
					                 \frac{A_{22}}{det(A)}  & \frac{-A_{12}}{det(A)} \\
					                 \frac{-A_{21}}{det(A)} & \frac{A_{11}}{det(A)}
				                 \end{bmatrix}
			      \end{split}
		      \]
	\end{itemize}
\end{homeworkProblem}

% Homework problem 3 sec 4.2 problem 4
\begin{homeworkProblem}
	Find the value of $k$ such that satisfies the following equation.
	\[
		det \begin{bmatrix}
			b_1 + c_1 & b_2 + c_2 & b_3 + c_3 \\
			a_1 + c_1 & a_2 + c_2 & a_3 + c_3 \\
			a_1 + b_1 & a_2 + b_2 & a_3 + b_3 \\
		\end{bmatrix} = k \cdot det \begin{bmatrix}
			a_1 & a_2 & a_3 \\
			b_1 & b_2 & b_3 \\
			c_1 & c_2 & c_3 \\
		\end{bmatrix}
	\]
	\[
		\begin{split}
			det \begin{bmatrix}
				    b_1 + c_1 & b_2 + c_2 & b_3 + c_3 \\
				    a_1 + c_1 & a_2 + c_2 & a_3 + c_3 \\
				    a_1 + b_1 & a_2 + b_2 & a_3 + b_3 \\
			    \end{bmatrix} = 2a_1b_2c_3 + 2a_2b_3c_1 + 2a_3b_1c_2 - 2a_3b_2c_1 - 2a_2b_1c_3 - 2a_1b_3c_2 \\
			det \begin{bmatrix}
				    a_1 & a_2 & a_3 \\
				    b_1 & b_2 & b_3 \\
				    c_1 & c_2 & c_3 \\
			    \end{bmatrix} = a_1b_2c_3 + a_2b_3c_1 + a_3b_1c_2 - a_3b_2c_1 - a_2b_1c_3 - a_1b_3c_2       \\
			k = 2
		\end{split}
	\]
\end{homeworkProblem}

% Homework problem 4 sec 4.2 problem 12
\begin{homeworkProblem}
	Evaluate the determinant of the given matrix by cofactor expansion along the indicated row.
	\[
		\begin{bmatrix}
			1  & -1 & 2  & -1 \\
			-3 & 4  & 1  & -1 \\
			2  & -5 & -3 & 8  \\
			-2 & 6  & -4 & 1  \\
		\end{bmatrix}
	\]
	along the fourth row.
	\[
		\begin{split}
			(-1)^{4+1} \cdot -2 \cdot \begin{vmatrix}
				                          -1 & 2  & -1 \\
				                          4  & 1  & -1 \\
				                          -5 & -3 & 8
			                          \end{vmatrix} +
			(-1)^{4+2} \cdot 6 \cdot \begin{vmatrix}
				                         1  & 2  & -1 \\
				                         -3 & 1  & -1 \\
				                         2  & -3 & 8
			                         \end{vmatrix}                                                                                                          \\
			+ (-1)^{4+3} \cdot -4 \cdot \begin{vmatrix}
				                            1  & -1 & 2  \\
				                            -3 & 4  & 1  \\
				                            2  & -5 & -3
			                            \end{vmatrix} + (-1)^{4+4} \cdot 1 \cdot \begin{vmatrix}
				                                                                     1  & -1 & 2  \\
				                                                                     -3 & 4  & 1  \\
				                                                                     2  & -5 & -3
			                                                                     \end{vmatrix}                                                              \\
			2 \cdot \left((-1)^{1+1} \cdot -1 \cdot \begin{vmatrix}
				                                        1  & -1 \\
				                                        -3 & 4
			                                        \end{vmatrix} + (-1)^{1+2} \cdot 2 \cdot \begin{vmatrix}
				                                                                                 4  & -1 \\
				                                                                                 -5 & 8
			                                                                                 \end{vmatrix} + (-1)^{1+3} \cdot -1 \cdot \begin{vmatrix}
				                                                                                                                           4  & 1  \\
				                                                                                                                           -5 & -3
			                                                                                                                           \end{vmatrix}\right)  \\
			+ 6 \cdot \left((-1)^{2+1} \cdot 1 \cdot \begin{vmatrix}
				                                         -3 & 1  \\
				                                         2  & -3
			                                         \end{vmatrix} + (-1)^{2+2} \cdot 2 \cdot \begin{vmatrix}
				                                                                                  1 & -1 \\
				                                                                                  2 & 8
			                                                                                  \end{vmatrix} + (-1)^{2+3} \cdot -1 \cdot \begin{vmatrix}
				                                                                                                                            1 & -1 \\
				                                                                                                                            2 & -5
			                                                                                                                            \end{vmatrix}\right) \\
			-4 \cdot \left((-1)^{3+1} \cdot 1 \cdot \begin{vmatrix}
				                                        -3 & 4  \\
				                                        -5 & -3
			                                        \end{vmatrix} + (-1)^{3+2} \cdot 2 \cdot \begin{vmatrix}
				                                                                                 1 & 4  \\
				                                                                                 2 & -3
			                                                                                 \end{vmatrix} + (-1)^{3+3} \cdot -1 \cdot \begin{vmatrix}
				                                                                                                                           1 & -1 \\
				                                                                                                                           2 & -5
			                                                                                                                           \end{vmatrix}\right)  \\
			+ 1 \cdot \left((-1)^{4+1} \cdot 1 \cdot \begin{vmatrix}
				                                         -3 & 4  \\
				                                         -5 & -3
			                                         \end{vmatrix} + (-1)^{4+2} \cdot 2 \cdot \begin{vmatrix}
				                                                                                  1 & 4  \\
				                                                                                  2 & -3
			                                                                                  \end{vmatrix} + (-1)^{4+3} \cdot -1 \cdot \begin{vmatrix}
				                                                                                                                            1 & -1 \\
				                                                                                                                            2 & -5
			                                                                                                                            \end{vmatrix}\right) \\
			2((-1)^{1+1} \cdot -1 \cdot (4 - 3) + (-1)^{1+2} \cdot 2 \cdot (4 + 15) + (-1)^{1+3} \cdot -1 \cdot (-12 - 5))                                   \\
			+ 6((-1)^{2+1} \cdot 1 \cdot (-9 + 2) + (-1)^{2+2} \cdot 2 \cdot (-3 - 16) + (-1)^{2+3} \cdot -1 \cdot (-8 - 2))                                 \\
			- 4((-1)^{3+1} \cdot 1 \cdot (-9 + 20) + (-1)^{3+2} \cdot 2 \cdot (-3 - 8) + (-1)^{3+3} \cdot -1 \cdot (-4 - 10))                                \\
			+ 1((-1)^{4+1} \cdot 1 \cdot (-27 + 20) + (-1)^{4+2} \cdot 2 \cdot (-3 - 8) + (-1)^{4+3} \cdot -1 \cdot (-4 + 10))                               \\
			154
		\end{split}
	\]
\end{homeworkProblem}

% Homework problem 5 sec 4.2 problem 28
\begin{homeworkProblem}
	Compute $det(E_i)$ if $E_i$ is an elementary matrix of type $i$.
	\begin{itemize}
		\item $E_1$ is obtained by interchanging two rows of $I_n$.
		      \[
			      det(E_1) = -1
		      \]
		\item $E_2$ is obtained by multiplying a row of $I_n$ by a nonzero scalar.
		      \[
			      det(E_2) = k
		      \]
		\item $E_3$ is obtained by adding a multiple of one row of $I_n$ to another row.
		      \[
			      det(E_3) = 1
		      \]
	\end{itemize}

\end{homeworkProblem}

% Homework problem 6 sec 4.3 problem 5
\begin{homeworkProblem}
	Use Cramer's rule to solve the given system of linear equations.
	\[
		\begin{aligned}
			x_1 - x_2 + 4x_3   & = -4 \\
			-8x_1 + 3x_2 + x_3 & = 8  \\
			2x_1 + x_2 + x_3   & = 0
		\end{aligned}
	\]
	\[
		\begin{split}
			det(A)   & = \begin{bmatrix}
				             1  & -1 & 4 \\
				             -8 & 3  & 1 \\
				             2  & 1  & 1
			             \end{bmatrix} = 32                                      \\
			det(A_1) & = \begin{bmatrix}
				             -4 & -1 & 4 \\
				             8  & 3  & 1 \\
				             0  & 1  & 1
			             \end{bmatrix} = 32                                      \\
			det(A_2) & = \begin{bmatrix}
				             1  & -4 & 4 \\
				             -8 & 8  & 1 \\
				             2  & 0  & 1
			             \end{bmatrix} = -96                                     \\
			det(A_3) & = \begin{bmatrix}
				             1  & -1 & -4 \\
				             -8 & 3  & 8  \\
				             2  & 1  & 0
			             \end{bmatrix} = 32                                      \\
			x_1      & = \frac{det(A_1)}{det(A)} = \frac{32}{-64} = -\frac{1}{2} \\
			x_2      & = \frac{det(A_2)}{det(A)} = \frac{-96}{-64} = \frac{3}{2} \\
			x_3      & = \frac{det(A_3)}{det(A)} = \frac{32}{-64} = -\frac{1}{2}
		\end{split}
	\]
\end{homeworkProblem}

% Homework problem 7 sec 4.3 problem 10
\begin{homeworkProblem}
	A matrix $M \in M_{nxn}(F)$ is called nipotent if, some positive interrger $k$, $M^k = 0$, where $0$ is the $n x n$ zero matrix. Prove that if $M$ is nilpotent, then $det(M) = 0$.
	\begin{proof}
		Let $M \in M_{nxn}(F)$ be a nilpotent matrix. Then there exists a positive integer $k$ such that $M^k = 0$. We will prove that $det(M) = 0$ using properties of determinants
		Given that $M^k = 0$, then $det(M^k) = det(0) = 0$.
		Using the property of determinants that $det(M^k) = det(M)^k$, then $det(M)^k = 0$.
		Since the only solution that satisfies is when $det(M) = 0$
		Therefore, if $M$ is nilpotent, then $det(M) = 0$
	\end{proof}
\end{homeworkProblem}

% Homework problem 8 sec 4.3 problem 16
\begin{homeworkProblem}
	Use determinants to prove that if $A,B \in M_{nxn}(F)$ are sucht that $AB = I$, then $A$ is invertible (and hence $B = A^{-1}$).
	\begin{proof}
		\begin{align*}
			\intertext{If $A$ is invertible, then}
			AB                         & = I                             \\
			\intertext{Given the only solution to the equation is when $B = A^{-1}$}
			\det(A) \cdot \det(A^{-1}) & = \det(AA^{-1}) = \det(I_n) = 1 \\
		\end{align*}
		Therefore, $\det(A^{-1}) = \frac{1}{\det(A)}$ if $A$ is invertible and $\det(A) \neq 0$ and $B = A^{-1}$.
	\end{proof}
\end{homeworkProblem}

% Homework problem 9 sec 4.3 problem 24
\begin{homeworkProblem}
	Let $A \in M_{nxn}(F)$ have the form
	\[
		A = \begin{bmatrix}
			0      & 0      & 0      & \cdots & 0      & a_{0}   \\
			-1     & 0      & 0      & \cdots & 0      & a_{1}   \\
			0      & -1     & 0      & \cdots & 0      & a_{2}   \\
			\vdots & \vdots & \vdots & \ddots & \vdots & \vdots  \\
			0      & 0      & 0      & \cdots & -1     & a_{n-1}
		\end{bmatrix}
	\]

	Compute $det(A + tI)$ where $I$ is the $n x n$ identity matrix.
	\[
		\begin{split}
			A + tI = \begin{vmatrix}
				         t      & 0      & 0      & \cdots & 0      & a_{0}    \\
				         -1     & t      & 0      & \cdots & 0      & a_{1}    \\
				         0      & -1     & t      & \cdots & 0      & a_{2}    \\
				         \vdots & \vdots & \vdots & \ddots & \vdots & \vdots   \\
				         0      & 0      & 0      & \cdots & -1     & ta_{n-1} \\
			         \end{vmatrix} \\
			det(A_{2x2}) = \begin{vmatrix}
				               t  & 0 \\
				               -1 & t
			               \end{vmatrix} = t^2                             \\
			det(A_{3x3}) = \begin{vmatrix}
				               t  & 0  & 0 \\
				               -1 & t  & 0 \\
				               0  & -1 & t
			               \end{vmatrix} = t^3 + t + 1                     \\
			det(A_{nxn}) = t^n + t^{n-2} + t^{n-3} + \ldots + t + 1
		\end{split}
	\]
\end{homeworkProblem}

% Homework problem 10 sec 4.4 Section 4.4 problem 4th
\begin{homeworkProblem}
\end{homeworkProblem}



\end{document}
