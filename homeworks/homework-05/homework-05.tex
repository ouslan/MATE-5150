\documentclass{article}

\usepackage{fancyhdr}
\usepackage{extramarks}
\usepackage{amsmath}
\usepackage{amsthm}
\usepackage{amsfonts}
\usepackage{tikz}
\usepackage[plain]{algorithm}
\usepackage{algpseudocode}

\usetikzlibrary{automata,positioning}

%
% Basic Document Settings
%

\topmargin=-0.45in
\evensidemargin=0in
\oddsidemargin=0in
\textwidth=6.5in
\textheight=9.0in
\headsep=0.25in

\linespread{1.1}

\pagestyle{fancy}
\lhead{\hmwkAuthorName}
\chead{\hmwkClass\ (\hmwkClassInstructor): \hmwkTitle}
\rhead{\firstxmark}
\lfoot{\lastxmark}
\cfoot{\thepage}

\renewcommand\headrulewidth{0.4pt}
\renewcommand\footrulewidth{0.4pt}

\setlength\parindent{0pt}

%
% Create Problem Sections
%

\newcommand{\enterProblemHeader}[1]{
	\nobreak\extramarks{}{Problem \arabic{#1} continued on next page\ldots}\nobreak{}
	\nobreak\extramarks{Problem \arabic{#1} (continued)}{Problem \arabic{#1} continued on next page\ldots}\nobreak{}
}

\newcommand{\exitProblemHeader}[1]{
	\nobreak\extramarks{Problem \arabic{#1} (continued)}{Problem \arabic{#1} continued on next page\ldots}\nobreak{}
	\stepcounter{#1}
	\nobreak\extramarks{Problem \arabic{#1}}{}\nobreak{}
}

\setcounter{secnumdepth}{0}
\newcounter{partCounter}
\newcounter{homeworkProblemCounter}
\setcounter{homeworkProblemCounter}{1}
\nobreak\extramarks{Problem \arabic{homeworkProblemCounter}}{}\nobreak{}

%
% Homework Problem Environment
%
% This environment takes an optional argument. When given, it will adjust the
% problem counter. This is useful for when the problems given for your
% assignment aren't sequential. See the last 3 problems of this template for an
% example.
%
\newenvironment{homeworkProblem}[1][-1]{
	\ifnum#1>0
		\setcounter{homeworkProblemCounter}{#1}
	\fi
	\section{Problem \arabic{homeworkProblemCounter}}
	\setcounter{partCounter}{1}
	\enterProblemHeader{homeworkProblemCounter}
}{
	\exitProblemHeader{homeworkProblemCounter}
}

%
% Homework Details
%   - Title
%   - Due date
%   - Class
%   - Section/Time
%   - Instructor
%   - Author
%

\newcommand{\hmwkTitle}{Asignacion\ \#3}
\newcommand{\hmwkDueDate}{Septiembre 26, 2024}
\newcommand{\hmwkClass}{MATE 5150}
\newcommand{\hmwkClassInstructor}{Dr. Pedro Vasquez}
\newcommand{\hmwkAuthorName}{\textbf{Alejandro Ouslan}}

%
% Title Page
%

\title{
	\vspace{2in}
	\textmd{\textbf{\hmwkClass:\ \hmwkTitle}}\\
	\normalsize\vspace{0.1in}\small{Due\ on\ \hmwkDueDate}\\
	\vspace{0.1in}\large{\textit{\hmwkClassInstructor}}
	\vspace{3in}
}

\author{\hmwkAuthorName}
\date{}

\renewcommand{\part}[1]{\textbf{\large Part \Alph{partCounter}}\stepcounter{partCounter}\\}

%
% Various Helper Commands
%

% Useful for algorithms
\newcommand{\alg}[1]{\textsc{\bfseries \footnotesize #1}}

% For derivatives
\newcommand{\deriv}[1]{\frac{\mathrm{d}}{\mathrm{d}x} (#1)}

% For partial derivatives
\newcommand{\pderiv}[2]{\frac{\partial}{\partial #1} (#2)}

% Integral dx
\newcommand{\dx}{\mathrm{d}x}

% Alias for the Solution section header
\newcommand{\solution}{\textbf{\large Solution}}

% Probability commands: Expectation, Variance, Covariance, Bias
\newcommand{\E}{\mathrm{E}}
\newcommand{\Var}{\mathrm{Var}}
\newcommand{\Cov}{\mathrm{Cov}}
\newcommand{\Bias}{\mathrm{Bias}}

\begin{document}

\maketitle

\pagebreak

% Homework problem 1 sec 3.1 problem 3b
\begin{homeworkProblem}
	Use the proof of Theorem 3.2 to obtain the inverse of each of the following matrices:
	$$B = \begin{bmatrix} 1 & 0 & 0 \\ 0 & 3 & 0 \\ 0 & 0 & 1 \end{bmatrix}$$
\end{homeworkProblem}

% Homework problem 2 sec 3.1 problem 6
\begin{homeworkProblem}
	Let $A$ be an $m \times n$ matrix. Prove that if $B$ can be obtained from $A$ by an elementary row [column] operation, then $B^T$ can be obtained from $A^T$ by the
	corresponding elementary column [row] operation.
\end{homeworkProblem}

% Homework problem 3 sec 3.1 problem 10
\begin{homeworkProblem}
	Prove that any elementary row [column] operation of type 2 can be obtained by dividing some row [column] by a nonzero scalar.
\end{homeworkProblem}

% Homework problem 4 sec 3.2 problem 2g
\begin{homeworkProblem}
	Find the rank of the following matrix:
	$$G = \begin{bmatrix} 1 & 1 & 0 & 1 \\ 2 & 2 & 0 & 2 \\ 1 & 1 & 0 & 0 \\ 1 & 1 & 0 & 1 \end{bmatrix}$$
\end{homeworkProblem}

% Homework problem 5 sec 3.2 problem 5f
\begin{homeworkProblem}
	For each of the following matrices, compute the rank and the inverse if exists:
	$$ F = \begin{bmatrix} 1 & 2 & 1 \\ 1 & 0 & 1 \\ 1 & 1 & 1 \end{bmatrix} $$
\end{homeworkProblem}

% Homework problem 6 sec 3.2 problem 6b
\begin{homeworkProblem}
	For each of the following linear transformations $T$, determine whether $T$ is invertible, and compute $T^{-1}$ if it exists:
	$$ T = P_2 (\mathbb{R}) \to P_2 (\mathbb{R}) \quad \text{defined by} \quad T(f(x)) = (x+1)f'(x)$$
\end{homeworkProblem}

% Homework problem 7 sec 3.2 problem 6c
\begin{homeworkProblem}
	For each of the following linear transformations $T$, determine whether $T$ is invertible, and compute $T^{-1}$ if it exists:
	$$ T = \mathbb{R}^3 \to \mathbb{R}^3 \quad \text{defined by} \quad T(a_1, a_2, a_3) = (a_1 + 2a_2 + 3a_3, -a_1 + a_2 + 2a_3, a_1 + a_3)$$
\end{homeworkProblem}

% Homework problem 8 sec 3.2 problem 14
\begin{homeworkProblem}
	Let $T, U: V \to W$ be linear transformations.
	\begin{itemize}
		\item Prove that $R(T + U) \subseteq R(T) + R(U)$.
		\item Prove that if $W$ if finite-dimensional, then $rank(T + U) \leq rank(T) + rank(U)$.
		\item Deduce from (b) that if $rank(A + B) \leq rank(A) + rank(B)$ for any $m \times n$ matrices $A$ and $B$.
	\end{itemize}
\end{homeworkProblem}


\end{document}
