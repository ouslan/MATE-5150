\documentclass{article}

\usepackage{fancyhdr}
\usepackage{extramarks}
\usepackage{amsmath}
\usepackage{amsthm}
\usepackage{amsfonts}
\usepackage{tikz}
\usepackage[plain]{algorithm}
\usepackage{algpseudocode}

\usetikzlibrary{automata,positioning}

%
% Basic Document Settings
%

\topmargin=-0.45in
\evensidemargin=0in
\oddsidemargin=0in
\textwidth=6.5in
\textheight=9.0in
\headsep=0.25in

\linespread{1.1}

\pagestyle{fancy}
\lhead{\hmwkAuthorName}
\chead{\hmwkClass\ (\hmwkClassInstructor): \hmwkTitle}
\rhead{\firstxmark}
\lfoot{\lastxmark}
\cfoot{\thepage}

\renewcommand\headrulewidth{0.4pt}
\renewcommand\footrulewidth{0.4pt}

\setlength\parindent{0pt}

%
% Create Problem Sections
%

\newcommand{\enterProblemHeader}[1]{
	\nobreak\extramarks{}{Problem \arabic{#1} continued on next page\ldots}\nobreak{}
	\nobreak\extramarks{Problem \arabic{#1} (continued)}{Problem \arabic{#1} continued on next page\ldots}\nobreak{}
}

\newcommand{\exitProblemHeader}[1]{
	\nobreak\extramarks{Problem \arabic{#1} (continued)}{Problem \arabic{#1} continued on next page\ldots}\nobreak{}
	\stepcounter{#1}
	\nobreak\extramarks{Problem \arabic{#1}}{}\nobreak{}
}

\setcounter{secnumdepth}{0}
\newcounter{partCounter}
\newcounter{homeworkProblemCounter}
\setcounter{homeworkProblemCounter}{1}
\nobreak\extramarks{Problem \arabic{homeworkProblemCounter}}{}\nobreak{}

%
% Homework Problem Environment
%
% This environment takes an optional argument. When given, it will adjust the
% problem counter. This is useful for when the problems given for your
% assignment aren't sequential. See the last 3 problems of this template for an
% example.
%
\newenvironment{homeworkProblem}[1][-1]{
	\ifnum#1>0
		\setcounter{homeworkProblemCounter}{#1}
	\fi
	\section{Problem \arabic{homeworkProblemCounter}}
	\setcounter{partCounter}{1}
	\enterProblemHeader{homeworkProblemCounter}
}{
	\exitProblemHeader{homeworkProblemCounter}
}

%
% Homework Details
%   - Title
%   - Due date
%   - Class
%   - Section/Time
%   - Instructor
%   - Author
%

\newcommand{\hmwkTitle}{Asignacion\ \#1}
\newcommand{\hmwkDueDate}{Septiembre 5, 2024}
\newcommand{\hmwkClass}{MATE 5150}
\newcommand{\hmwkClassInstructor}{Dr. Pedro Vasquez}
\newcommand{\hmwkAuthorName}{\textbf{Alejandro Ouslan}}

%
% Title Page
%

\title{
	\vspace{2in}
	\textmd{\textbf{\hmwkClass:\ \hmwkTitle}}\\
	\normalsize\vspace{0.1in}\small{Due\ on\ \hmwkDueDate}\\
	\vspace{0.1in}\large{\textit{\hmwkClassInstructor}}
	\vspace{3in}
}

\author{\hmwkAuthorName}
\date{}

\renewcommand{\part}[1]{\textbf{\large Part \Alph{partCounter}}\stepcounter{partCounter}\\}

%
% Various Helper Commands
%

% Useful for algorithms
\newcommand{\alg}[1]{\textsc{\bfseries \footnotesize #1}}

% For derivatives
\newcommand{\deriv}[1]{\frac{\mathrm{d}}{\mathrm{d}x} (#1)}

% For partial derivatives
\newcommand{\pderiv}[2]{\frac{\partial}{\partial #1} (#2)}

% Integral dx
\newcommand{\dx}{\mathrm{d}x}

% Alias for the Solution section header
\newcommand{\solution}{\textbf{\large Solution}}

% Probability commands: Expectation, Variance, Covariance, Bias
\newcommand{\E}{\mathrm{E}}
\newcommand{\Var}{\mathrm{Var}}
\newcommand{\Cov}{\mathrm{Cov}}
\newcommand{\Bias}{\mathrm{Bias}}

\begin{document}

\maketitle

\pagebreak

% Homework sec 1.4, prob 2d
\begin{homeworkProblem}
	Solve the following systems of linear equations by the method introduced introduces in this section.
	\[
		\begin{aligned}
			x_1 & \quad + 2x_2 \quad & + 2x_3 \quad & \quad \quad & = 2  \\
			x_1 & \quad \quad        & + 8x_3 \quad & + 5x_4      & = -6 \\
			x_1 & \quad + x_2 \quad  & + 5x_3 \quad & + 5x_4      & = 3
		\end{aligned}
	\]
\end{homeworkProblem}

% Homework sec 1.4, prob 5d
\begin{homeworkProblem}
	In each part, determine whether the given vector is in the span of $S$.
	\[
		(2,-1,1,-3) \quad \text{where} \quad S = \{ x^3 + x^2 + x + 1, x^2 + x + 1, x + 1\}
	\]
\end{homeworkProblem}

% Homework sec 1.4, prob 6
\begin{homeworkProblem}
	Show that the vectors $(1,1,0)$, $(1,0,1)$, and $(0,1,1)$ generate $\mathbb{F}^3$
\end{homeworkProblem}

% Homework sec 1.4, prob 15
\begin{homeworkProblem}
	Let $S_1$ and $S_2$ be subsets of a vector space $V$. Prove that $span(S_1 \cap S_2) \subseteq span(S_1) \cap span(S_2)$. Give an example in which $span(S_1 \cap S_2) $ and
	$span(S_1) \cap span(S_2)$ are equal and one in which they are not equal.
\end{homeworkProblem}

% Homework sec 1.5, prob 2d
\begin{homeworkProblem}
	Determine whether the following sets are linearly dependent or linearly independent.
	$$\{ x^3 - x, 2x^2 + 4, -2x^3 + 3x^2 + 2x + 6 \}$$
\end{homeworkProblem}

% Homework sec 1.5, prob 6
\begin{homeworkProblem}
	In $M_{m \times n}(\mathbb{F})$, let $E_{ij}$ denote the matrix whose only nonzero entry is a 1 in the $i$th row and $j$th column. Prove that $\{ E_{ij} : 1 \leq i \leq m, 1 \leq j \leq n \}$
	is linearly independent.
\end{homeworkProblem}

% Homework sec 1.5, prob 20
\begin{homeworkProblem}
	Let $f,g, \in \mathcal{F}(R,R)$ be the function defined by $f(x) = e^{rt}$ and $g(x) = e^{st}$, where $r \neq s$. Prove that $f$ and $g$ are linearly independent in $\mathcal{F}(R,R)$.
\end{homeworkProblem}

% Homework sec 1.6, prob 2e
\begin{homeworkProblem}
	Determine which of the following sets are bases for $\mathbb{R}^3$.
	\[ \{ (1,-3,1), (-3,1,3), (-2,-10,2) \} \]
\end{homeworkProblem}

% Homework sec 1.6, prob 7
\begin{homeworkProblem}
	The vectors $u_1 = (2,-3,1)$, $u_2 = (1,4,-2)$, $u_3 = (-8,12,-4)$, $u_4 = (1,37,-17)$ and $u_5 = (-3,-5,8)$ generate $\mathbb{R}^3$. Find a subset of the set $\{ u_1, u_2, u_3, u_4, u_5 \}$
	that is a basis for $\mathbb{R}^3$.
\end{homeworkProblem}

% Homework sec 1.6, prob 9
\begin{homeworkProblem}
	The vectors $u_1 = (1,1,1,1)$, $u_2 = (0,1,1,1)$, $u_3 = (0,0,1,1)$, and $u_4 = (0,0,0,1)$ form a basis for $\mathbb{R}^4$. Find the unique representation of an  arbitrary vector $v = (a_1, a_2, a_3, a_4)$
	$\mathbb{R}^4$ as a linear combination of $u_1, u_2, u_3, u_4$.
\end{homeworkProblem}

% Homework sec 1.6, prob 10
\begin{homeworkProblem}
	In each part, use the Lagrange interpolation formula to construct the polynomial of smallest degree whose graph contains the following points. $(-2,3),(-1,-6),(1,0),(3,-2)$.
\end{homeworkProblem}

% Homework sec 1.6, prob 16
\begin{homeworkProblem}
	The set of all upper triangular $n \times n$ matrices is a subspace of $W$ of $M_{n \times n}(\mathbb{F})$. Find a basis for $W$. What is the dimension of $W$?
\end{homeworkProblem}


\end{document}


