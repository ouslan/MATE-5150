\documentclass{article}

\usepackage{fancyhdr}
\usepackage{extramarks}
\usepackage{amsmath}
\usepackage{amsthm}
\usepackage{amsfonts}
\usepackage{tikz}
\usepackage[plain]{algorithm}
\usepackage{algpseudocode}

\usetikzlibrary{automata,positioning}

%
% Basic Document Settings
%

\topmargin=-0.45in
\evensidemargin=0in
\oddsidemargin=0in
\textwidth=6.5in
\textheight=9.0in
\headsep=0.25in

\linespread{1.1}

\pagestyle{fancy}
\lhead{\hmwkAuthorName}
\chead{\hmwkClass\ (\hmwkClassInstructor): \hmwkTitle}
\rhead{\firstxmark}
\lfoot{\lastxmark}
\cfoot{\thepage}

\renewcommand\headrulewidth{0.4pt}
\renewcommand\footrulewidth{0.4pt}

\setlength\parindent{0pt}

%
% Create Problem Sections
%

\newcommand{\enterProblemHeader}[1]{
	\nobreak\extramarks{}{Problem \arabic{#1} continued on next page\ldots}\nobreak{}
	\nobreak\extramarks{Problem \arabic{#1} (continued)}{Problem \arabic{#1} continued on next page\ldots}\nobreak{}
}

\newcommand{\exitProblemHeader}[1]{
	\nobreak\extramarks{Problem \arabic{#1} (continued)}{Problem \arabic{#1} continued on next page\ldots}\nobreak{}
	\stepcounter{#1}
	\nobreak\extramarks{Problem \arabic{#1}}{}\nobreak{}
}

\setcounter{secnumdepth}{0}
\newcounter{partCounter}
\newcounter{homeworkProblemCounter}
\setcounter{homeworkProblemCounter}{1}
\nobreak\extramarks{Problem \arabic{homeworkProblemCounter}}{}\nobreak{}

%
% Homework Problem Environment
%
% This environment takes an optional argument. When given, it will adjust the
% problem counter. This is useful for when the problems given for your
% assignment aren't sequential. See the last 3 problems of this template for an
% example.
%
\newenvironment{homeworkProblem}[1][-1]{
	\ifnum#1>0
		\setcounter{homeworkProblemCounter}{#1}
	\fi
	\section{Problem \arabic{homeworkProblemCounter}}
	\setcounter{partCounter}{1}
	\enterProblemHeader{homeworkProblemCounter}
}{
	\exitProblemHeader{homeworkProblemCounter}
}

%
% Homework Details
%   - Title
%   - Due date
%   - Class
%   - Section/Time
%   - Instructor
%   - Author
%

\newcommand{\hmwkTitle}{Asignacion\ \#2}
\newcommand{\hmwkDueDate}{Septiembre 18, 2024}
\newcommand{\hmwkClass}{MATE 5150}
\newcommand{\hmwkClassInstructor}{Dr. Pedro Vasquez}
\newcommand{\hmwkAuthorName}{\textbf{Alejandro Ouslan}}

%
% Title Page
%

\title{
	\vspace{2in}
	\textmd{\textbf{\hmwkClass:\ \hmwkTitle}}\\
	\normalsize\vspace{0.1in}\small{Due\ on\ \hmwkDueDate}\\
	\vspace{0.1in}\large{\textit{\hmwkClassInstructor}}
	\vspace{3in}
}

\author{\hmwkAuthorName}
\date{}

\renewcommand{\part}[1]{\textbf{\large Part \Alph{partCounter}}\stepcounter{partCounter}\\}

%
% Various Helper Commands
%

% Useful for algorithms
\newcommand{\alg}[1]{\textsc{\bfseries \footnotesize #1}}

% For derivatives
\newcommand{\deriv}[1]{\frac{\mathrm{d}}{\mathrm{d}x} (#1)}

% For partial derivatives
\newcommand{\pderiv}[2]{\frac{\partial}{\partial #1} (#2)}

% Integral dx
\newcommand{\dx}{\mathrm{d}x}

% Alias for the Solution section header
\newcommand{\solution}{\textbf{\large Solution}}

% Probability commands: Expectation, Variance, Covariance, Bias
\newcommand{\E}{\mathrm{E}}
\newcommand{\Var}{\mathrm{Var}}
\newcommand{\Cov}{\mathrm{Cov}}
\newcommand{\Bias}{\mathrm{Bias}}

\begin{document}

\maketitle

\pagebreak

% Homework sec 1.4, prob 2d
\begin{homeworkProblem}
	Solve the following systems of linear equations by the method introduced introduces in this section.
	\[
		\begin{aligned}
			x_1 & \quad + 2x_2 \quad & + 2x_3 \quad & \quad \quad & = 2  \\
			x_1 & \quad \quad        & + 8x_3 \quad & + 5x_4      & = -6 \\
			x_1 & \quad + x_2 \quad  & + 5x_3 \quad & + 5x_4      & = 3
		\end{aligned}
	\]

	\[
		\begin{split}
			\begin{bmatrix}
				1 & 2 & 2 & 0 \\
				1 & 0 & 8 & 5 \\
				1 & 1 & 5 & 5
			\end{bmatrix}
			\begin{bmatrix}
				x_1 \\
				x_2 \\
				x_3 \\
				x_4
			\end{bmatrix} =
			\begin{bmatrix}
				2  \\
				-6 \\
				3
			\end{bmatrix} \\
			\begin{bmatrix}
				1 & 2 & 2 & 0 \\
				1 & 1 & 5 & 5 \\
				1 & 0 & 8 & 5
			\end{bmatrix}
			\begin{bmatrix}
				x_1 \\
				x_2 \\
				x_3 \\
				x_4
			\end{bmatrix} =
			\begin{bmatrix}
				2 \\
				3 \\
				-6
			\end{bmatrix} \\
			\begin{bmatrix}
				1 & 2  & 2 & 0 \\
				0 & -1 & 3 & 5 \\
				0 & -2 & 6 & 5
			\end{bmatrix}
			\begin{bmatrix}
				x_1 \\
				x_2 \\
				x_3 \\
				x_4
			\end{bmatrix} =
			\begin{bmatrix}
				2 \\
				1 \\
				-8
			\end{bmatrix} \\
			\begin{bmatrix}
				1 & 2 & 2  & 0  \\
				0 & 1 & -3 & -5 \\
				0 & 0 & 0  & -5
			\end{bmatrix}
			\begin{bmatrix}
				x_1 \\
				x_2 \\
				x_3 \\
				x_4
			\end{bmatrix} =
			\begin{bmatrix}
				2  \\
				-1 \\
				-10
			\end{bmatrix} \\
			\begin{bmatrix}
				1 & 0 & 8  & 20 \\
				0 & 1 & -3 & -5 \\
				0 & 0 & 0  & 1
			\end{bmatrix}
			\begin{bmatrix}
				x_1 \\
				x_2 \\
				x_3 \\
				x_4
			\end{bmatrix} =
			\begin{bmatrix}
				0  \\
				-1 \\
				2
			\end{bmatrix} \\
			\begin{bmatrix}
				1 & 0 & 8  & 0 \\
				0 & 1 & -3 & 0 \\
				0 & 0 & 0  & 1
			\end{bmatrix}
			\begin{bmatrix}
				x_1 \\
				x_2 \\
				x_3 \\
				x_4
			\end{bmatrix} =
			\begin{bmatrix}
				-20 \\
				9   \\
				2
			\end{bmatrix} \\
			(x_1, x_2, x_3, x_4) = (-20, 9, 0, 2)
		\end{split}
	\]
\end{homeworkProblem}

% Homework sec 1.4, prob 5g
\begin{homeworkProblem}
	In each part, determine whether the given vector is in the span of $S$.(change is wrohng)
	\[
		\begin{pmatrix}
			1  & 2 \\
			-3 & 4
		\end{pmatrix} \text{, } S = \left\{ \begin{pmatrix}
			1  & 0 \\
			-1 & 0
		\end{pmatrix},
		\begin{pmatrix}
			0 & 1 \\
			0 & 1
		\end{pmatrix},
		\begin{pmatrix}
			1 & 1 \\
			0 & 0
		\end{pmatrix} \right\}
	\]

	\[
		\begin{split}
			S &= a \begin{pmatrix}
				1  & 0 \\
				-1 & 0
			\end{pmatrix} + b \begin{pmatrix}
				0 & 1 \\
				0 & 1
			\end{pmatrix} + c \begin{pmatrix}
				1 & 1 \\
				0 & 0
			\end{pmatrix} \\
			S &= \begin{pmatrix}
				a  & 0 \\
				-a & 0
			\end{pmatrix} + \begin{pmatrix}
				0 & b \\
				0 & b
			\end{pmatrix} + \begin{pmatrix}
				c & c \\
				0 & 0
			\end{pmatrix} \\
			S &= \begin{pmatrix}
				a + c & b + c \\
				-a    & b
			\end{pmatrix}
		\end{split}
	\]
	Given $a = 3$, $b = 4$, and $c = -2$ we have that the given vector is in the span of $S$.
\end{homeworkProblem}

% Homework sec 1.4, prob 6
\begin{homeworkProblem}
	Show that the vectors $(1,1,0)$, $(1,0,1)$, and $(0,1,1)$ generate $\mathbb{F}^3$
	\begin{proof}
		\begin{align*}
			\text{Let } v                            & = (x, y, z) \in \mathbb{F}^3     \\
			\text{Then } v                           & = a(1,1,0) + b(1,0,1) + c(0,1,1) \\
			\text{Then } v                           & = \begin{bmatrix}
				                                             1 & 1 & 0 \\
				                                             1 & 0 & 1 \\
				                                             0 & 1 & 1
			                                             \end{bmatrix}
			\begin{bmatrix}
				a \\
				b \\
				c
			\end{bmatrix} = \begin{bmatrix}
				                x \\
				                y \\
				                z
			                \end{bmatrix}                                              \\
			\text{Then determinate } \begin{vmatrix}
				                         1 & 1 & 0 \\
				                         1 & 0 & 1 \\
				                         0 & 1 & 1
			                         \end{vmatrix} & = 1(0 - 1) - 1(1 - 0) = -1 \neq 0  \\
		\end{align*}
		Thus the vectors $(1,1,0)$, $(1,0,1)$, and $(0,1,1)$ generate $\mathbb{F}^3$.
	\end{proof}
\end{homeworkProblem}

% Homework sec 1.4, prob 15
\begin{homeworkProblem}
	Let $S_1$ and $S_2$ be subsets of a vector space $V$. Prove that $span(S_1 \cap S_2) \subseteq span(S_1) \cap span(S_2)$. Give an example in which $span(S_1 \cap S_2) $ and
	$span(S_1) \cap span(S_2)$ are equal and one in which they are not equal.
	\begin{proof}
		\begin{align*}
			\text{Let } x   & \in span(S_1 \cap S_2)                                                                  \\
			\text{Then } x  & = c_1v_1 + c_2v_2 + \ldots + c_nv_n \text{ where } v_i \in S_1 \cap S_2                 \\
			\text{Since } x & \in S_1 \cap S_2 \implies v_i \in S_1 \text{ and } v_i \in S_2 \text{ for all } i       \\
			\text{Thus } x  & = c_1v_1 + c_2v_2 + \ldots + c_nv_n \text{ where } v_i \in S_1 \text{ and } v_i \in S_2 \\
		\end{align*}
		Therefore $x \in span(S_1) \cap span(S_2)$, thus $span(S_1 \cap S_2) \subseteq span(S_1) \cap span(S_2)$.
	\end{proof}
\end{homeworkProblem}

% Homework sec 1.5, prob 2g
\begin{homeworkProblem}
	Determine whether the following sets are linearly dependent or linearly independent.
	\[
		\left\{
		\begin{pmatrix}
			1  & 0 \\
			-2 & 1
		\end{pmatrix},
		\begin{pmatrix}
			0 & -1 \\
			1 & 1
		\end{pmatrix},
		\begin{pmatrix}
			-1 & 2 \\
			1  & 0
		\end{pmatrix},
		\begin{pmatrix}
			2  & 1 \\
			-4 & 4
		\end{pmatrix}
		\right\}
	\]
	\[
		\begin{split}
			v = c_1 \begin{pmatrix}
				1  & 0 \\
				-2 & 1
			\end{pmatrix} +
			c_2 \begin{pmatrix}
				0 & -1 \\
				1 & 1
			\end{pmatrix} +
			c_3 \begin{pmatrix}
				-1 & 2 \\
				1  & 0
			\end{pmatrix} +
			c_4 \begin{pmatrix}
				2  & 1 \\
				-4 & 4
			\end{pmatrix} = \begin{pmatrix}
				0 & 0 \\
				0 & 0
			\end{pmatrix}
		\end{split}
	\]
	\[
		\begin{aligned}
			c_1 - c_3 + 2c_4         & = 0 \quad \text{(1)} \\
			-c_2 + 2c_3 + c_4        & = 0 \quad \text{(2)} \\
			-2c_1 + c_2 + c_3 - 4c_4 & = 0 \quad \text{(3)} \\
			c_1 + c_2 + 4c_4         & = 0 \quad \text{(4)}
		\end{aligned}
	\]
\end{homeworkProblem}

% Homework sec 1.5, prob 6
\begin{homeworkProblem}
	In \( M_{m \times n}(\mathbb{F}) \), let \( E_{ij} \) denote the matrix whose only nonzero entry is a 1 in the \( i \)th row and \( j \)th column. Prove that \( \{ E_{ij} : 1 \leq i \leq m, 1 \leq j \leq n \} \) is linearly independent.

	\[
		\sum_{i=1}^{m} \sum_{j=1}^{n} c_{ij} E_{ij} = 0 \quad \text{for scalars } c_{ij} \in \mathbb{F}
	\]
	\[
		\Rightarrow \quad \sum_{i=1}^{m} \sum_{j=1}^{n} c_{ij} E_{ij} = \begin{pmatrix}
			0      & 0      & \cdots & 0      \\
			0      & 0      & \cdots & 0      \\
			\vdots & \vdots & \ddots & \vdots \\
			0      & 0      & \cdots & 0
		\end{pmatrix}
	\]
	\[
		\Rightarrow \quad c_{ij} = 0 \quad \text{for all } i, j
	\]
\end{homeworkProblem}

% Homework sec 1.5, prob 20
\begin{homeworkProblem}
	Let \( f, g \in \mathcal{F}(\mathbb{R}, \mathbb{R}) \) be the functions defined by \( f(x) = e^{rx} \) and \( g(x) = e^{sx} \), where \( r \neq s \). Prove that \( f \) and \( g \) are linearly independent in \( \mathcal{F}(\mathbb{R}, \mathbb{R}) \).

	\[
		c_1 f(x) + c_2 g(x) = 0 \quad \text{for all } x \in \mathbb{R}
	\]
	\[
		\Rightarrow \quad c_1 e^{rx} + c_2 e^{sx} = 0
	\]
	\[
		\Rightarrow \quad e^{rx} (c_1 + c_2 e^{(s-r)x}) = 0
	\]
	\[
		\Rightarrow \quad c_1 + c_2 e^{(s-r)x} = 0 \quad \text{for all } x
	\]
	\[
		\Rightarrow \quad c_1 = 0 \quad \text{and} \quad c_2 = 0
	\]
\end{homeworkProblem}

% Homework sec 1.6, prob 2e
\begin{homeworkProblem}
	Determine which of the following sets are bases for \( \mathbb{R}^3 \).
	\[ \{ (1,-3,1), (-3,1,3), (-2,-10,2) \} \]

	Construct the matrix:
	\[
		A = \begin{pmatrix}
			1  & -3  & 1 \\
			-3 & 1   & 3 \\
			-2 & -10 & 2
		\end{pmatrix}
	\]

	Row reduce:
	\[
		\begin{pmatrix}
			1 & -3 & 1 \\
			0 & -8 & 6 \\
			0 & 0  & 0
		\end{pmatrix}
	\]

	Rank = 2.

	The set \( \{ (1,-3,1), (-3,1,3), (-2,-10,2) \} \) is \textbf{not a basis} for \( \mathbb{R}^3 \).
\end{homeworkProblem}
% Homework sec 1.6, prob 7
\begin{homeworkProblem}
	To find a basis for \(\mathbb{R}^3\) from the set \(\{ u_1, u_2, u_3, u_4, u_5 \}\), we can examine the linear independence of the vectors.

	We will form a matrix \(A\) with the vectors as rows and perform row reduction:

	\[
		A = \begin{bmatrix}
			2  & -3 & 1   \\
			1  & 4  & -2  \\
			-8 & 12 & -4  \\
			1  & 37 & -17 \\
			-3 & -5 & 8
		\end{bmatrix}
	\]

	Performing row reduction, we find:

	\[
		\begin{bmatrix}
			1 & 4 & -2           \\
			0 & 1 & -\frac{1}{5} \\
			0 & 0 & 0            \\
			0 & 0 & 0            \\
			0 & 0 & 0
		\end{bmatrix}
	\]

	From this row echelon form, we see that \(u_1\) and \(u_2\) are linearly independent and span \(\mathbb{R}^3\) with an additional vector.

	We can choose:

	\[
		\{ u_1, u_2, u_5 \}
	\]

	Thus, a basis for \(\mathbb{R}^3\) is given by:

	\[
		\{ (2, -3, 1), (1, 4, -2), (-3, -5, 8) \}
	\]
\end{homeworkProblem}

% Homework sec 1.6, prob 9
\begin{homeworkProblem}
	To express the vector \( v = (a_1, a_2, a_3, a_4) \) as a linear combination of the basis vectors \( u_1, u_2, u_3, u_4 \):

	\[
		v = c_1 u_1 + c_2 u_2 + c_3 u_3 + c_4 u_4
	\]

	From the equations:

	1. \( c_1 = a_1 \)
	2. \( c_2 = a_2 - a_1 \)
	3. \( c_3 = a_3 - a_2 \)
	4. \( c_4 = a_4 - a_3 \)

	The unique representation is:

	\[
		v = a_1 u_1 + (a_2 - a_1) u_2 + (a_3 - a_2) u_3 + (a_4 - a_3) u_4
	\]
\end{homeworkProblem}

% Homework sec 1.6, prob 10c
\begin{homeworkProblem}
	For the points \((-2,3), (-1,-6), (1,0), (3,-2)\):

	\[
		P(x) = 3 \left(-\frac{(x+1)(x-1)(x-3)}{15}\right) - 6 \left(\frac{(x+2)(x-1)(x-3)}{12}\right) - 2 \left(\frac{(x+2)(x+1)(x-1)}{40}\right)
	\]

	For the points \((-2,-6), (-1,6), (1,0), (3,-2)\):

	\[
		P(x) = -6 \left(-\frac{(x+1)(x-1)(x-3)}{15}\right) + 6 \left(\frac{(x+2)(x-1)(x-3)}{12}\right) - 2 \left(\frac{(x+2)(x+1)(x-1)}{40}\right)
	\]

	The specific polynomials can be computed and simplified from these expressions.
\end{homeworkProblem}

% Homework sec 1.6, prob 16
\begin{homeworkProblem}
	The set of all upper triangular \( n \times n \) matrices can be expressed as:

	\[
		W = \left\{ A = \begin{pmatrix}
			a_{11} & a_{12} & \cdots & a_{1n} \\
			0      & a_{22} & \cdots & a_{2n} \\
			0      & 0      & \cdots & a_{nn}
		\end{pmatrix} : a_{ij} \in \mathbb{F}, \; i \leq j \right\}
	\]

	A basis for \( W \) consists of matrices that have a single entry of 1 in the upper triangular position and 0 elsewhere. Specifically, these matrices can be represented as follows:
	\begin{enumerate}
		\item \( E_{11} = \begin{pmatrix} 1 & 0 & \cdots & 0 \\ 0 & 0 & \cdots & 0 \\ \vdots & \vdots & \ddots & \vdots \\ 0 & 0 & \cdots & 0 \end{pmatrix} \)
		\item  \( E_{12} = \begin{pmatrix} 0 & 1 & 0 & \cdots & 0 \\ 0 & 0 & 0 & \cdots & 0 \\ \vdots & \vdots & \ddots & \vdots \\ 0 & 0 & 0 & \cdots & 0 \end{pmatrix} \)
		\item \( E_{13} = \begin{pmatrix} 0 & 0 & 1 & 0 & \cdots & 0 \\ 0 & 0 & 0 & \cdots & 0 \\ \vdots & \vdots & \ddots & \vdots \\ 0 & 0 & 0 & \cdots & 0 \end{pmatrix} \)
		\item \(\vdots\)
		\item \( E_{nn} = \begin{pmatrix} 0 & 0 & \cdots & 0 \\ 0 & 0 & \cdots & 0 \\ \vdots & \vdots & \ddots & \vdots \\ 0 & 0 & \cdots & 1 \end{pmatrix} \)
	\end{enumerate}
	The total number of these basis matrices corresponds to the number of entries in the upper triangular part of the matrix, which is:

	\[
		\text{Dimension of } W = \frac{n(n+1)}{2}
	\]

	Thus, the dimension of the subspace \( W \) of upper triangular \( n \times n \) matrices is:

	\[
		\boxed{\frac{n(n+1)}{2}}
	\]

	A basis for \( W \) consists of the matrices \( E_{ij} \) where \( 1 \leq i \leq j \leq n \).
\end{homeworkProblem}


\end{document}


