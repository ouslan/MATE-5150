\documentclass{article}

\usepackage{fancyhdr}
\usepackage{extramarks}
\usepackage{amsmath}
\usepackage{amsthm}
\usepackage{amsfonts}
\usepackage{tikz}
\usepackage[plain]{algorithm}
\usepackage{algpseudocode}

\usetikzlibrary{automata,positioning}

%
% Basic Document Settings
%

\topmargin=-0.45in
\evensidemargin=0in
\oddsidemargin=0in
\textwidth=6.5in
\textheight=9.0in
\headsep=0.25in

\linespread{1.1}

\pagestyle{fancy}
\lhead{\hmwkAuthorName}
\chead{\hmwkClass\ (\hmwkClassInstructor): \hmwkTitle}
\rhead{\firstxmark}
\lfoot{\lastxmark}
\cfoot{\thepage}

\renewcommand\headrulewidth{0.4pt}
\renewcommand\footrulewidth{0.4pt}

\setlength\parindent{0pt}

%
% Create Problem Sections
%

\newcommand{\enterProblemHeader}[1]{
	\nobreak\extramarks{}{Problem \arabic{#1} continued on next page\ldots}\nobreak{}
	\nobreak\extramarks{Problem \arabic{#1} (continued)}{Problem \arabic{#1} continued on next page\ldots}\nobreak{}
}

\newcommand{\exitProblemHeader}[1]{
	\nobreak\extramarks{Problem \arabic{#1} (continued)}{Problem \arabic{#1} continued on next page\ldots}\nobreak{}
	\stepcounter{#1}
	\nobreak\extramarks{Problem \arabic{#1}}{}\nobreak{}
}

\setcounter{secnumdepth}{0}
\newcounter{partCounter}
\newcounter{homeworkProblemCounter}
\setcounter{homeworkProblemCounter}{1}
\nobreak\extramarks{Problem \arabic{homeworkProblemCounter}}{}\nobreak{}

%
% Homework Problem Environment
%
% This environment takes an optional argument. When given, it will adjust the
% problem counter. This is useful for when the problems given for your
% assignment aren't sequential. See the last 3 problems of this template for an
% example.
%
\newenvironment{homeworkProblem}[1][-1]{
	\ifnum#1>0
		\setcounter{homeworkProblemCounter}{#1}
	\fi
	\section{Problem \arabic{homeworkProblemCounter}}
	\setcounter{partCounter}{1}
	\enterProblemHeader{homeworkProblemCounter}
}{
	\exitProblemHeader{homeworkProblemCounter}
}

%
% Homework Details
%   - Title
%   - Due date
%   - Class
%   - Section/Time
%   - Instructor
%   - Author
%

\newcommand{\hmwkTitle}{Asignacion\ \#1}
\newcommand{\hmwkDueDate}{Septiembre 5, 2024}
\newcommand{\hmwkClass}{MATE 5150}
\newcommand{\hmwkClassInstructor}{Dr. Pedro Vasquez}
\newcommand{\hmwkAuthorName}{\textbf{Alejandro Ouslan}}

%
% Title Page
%

\title{
	\vspace{2in}
	\textmd{\textbf{\hmwkClass:\ \hmwkTitle}}\\
	\normalsize\vspace{0.1in}\small{Due\ on\ \hmwkDueDate}\\
	\vspace{0.1in}\large{\textit{\hmwkClassInstructor}}
	\vspace{3in}
}

\author{\hmwkAuthorName}
\date{}

\renewcommand{\part}[1]{\textbf{\large Part \Alph{partCounter}}\stepcounter{partCounter}\\}

%
% Various Helper Commands
%

% Useful for algorithms
\newcommand{\alg}[1]{\textsc{\bfseries \footnotesize #1}}

% For derivatives
\newcommand{\deriv}[1]{\frac{\mathrm{d}}{\mathrm{d}x} (#1)}

% For partial derivatives
\newcommand{\pderiv}[2]{\frac{\partial}{\partial #1} (#2)}

% Integral dx
\newcommand{\dx}{\mathrm{d}x}

% Alias for the Solution section header
\newcommand{\solution}{\textbf{\large Solution}}

% Probability commands: Expectation, Variance, Covariance, Bias
\newcommand{\E}{\mathrm{E}}
\newcommand{\Var}{\mathrm{Var}}
\newcommand{\Cov}{\mathrm{Cov}}
\newcommand{\Bias}{\mathrm{Bias}}

\begin{document}

\maketitle

\pagebreak

% Homework problem 1
\begin{homeworkProblem}
	Find the equation of the plane through he following pairs of points in space. $P_1(1,1,1)$, $P_2(5,5,5)$, and
	$P_3(-6,4,2)$.
	\[
		\begin{split}
			\overrightarrow{AB} &= (5-1, 5-1, 5-1) = (4,4,4) \\
			\overrightarrow{AC} &= (-6-1, 4-1, 2-1) = (-7,3,1) \\
			x &= (1,1,1) + s(4,4,4) + t(-7,3,1)
		\end{split}
	\]
\end{homeworkProblem}

% Homework problem 2
\begin{homeworkProblem}
	Show thaat the midpoint of the line ssegment joining the points $(a,b)$ and $(c,d)$ is ($\frac{a+c}{2}, \frac{b+d}{2}$).
	\[
		\begin{split}
			\text{Midpoint} &= \frac{1}{2}(\overrightarrow{OA} + \overrightarrow{OB}) \\
			&= \frac{1}{2}(\begin{bmatrix} a \\ b \end{bmatrix} + \begin{bmatrix} c \\ d \end{bmatrix}) \\
			&= \frac{1}{2}(\begin{bmatrix} a + c \\ b + d \end{bmatrix}) \\
			&= (\frac{a+c}{2}, \frac{b+d}{2})
		\end{split}
	\]
\end{homeworkProblem}

% Homework problem 3
\begin{homeworkProblem}
	Let $S = \{0,1\}$ and $F = R$. In $\mathcal{F(S,R)}$, show that $f= g$ and $f + g = h$, where $f(t) = 2t + 1$,
	$g(t) = 1 + 4t - 2t^2$, and $h(t) = 5^t + 1$. \\
	\textbf{For $f = g$}
	\[
		\begin{split}
			f(0) &= g(0) \\
			2t + 1 &= 1 + 4t - 2t^2 \\
			1 &= 1 \\
			f(1) &= g(1) \\
			2t + 1 &= 1 + 4t - 2t^2 \\
			3 &= 3 \\
		\end{split}
	\]

	\textbf{For $f + g = h$}
	\[
		\begin{split}
			f(0) + g(0) &= h(0) \\
			2t + 1 + 1 + 4t - 2t^2  &= 5^t + 1 \\
			1 + 1 &= 2 \\
			f(1) + g(1) &= h(1) \\
			2t + 1 + 1 + 4t - 2t^2  &= 5^t + 1 \\
			6 &= 6
		\end{split}
	\]
\end{homeworkProblem}

% Homework problem 4
\begin{homeworkProblem}
	Let $V$ denote the set of ordered pairs of real numbers. If $(a_1, a_2)$ and $(b_1, b_2)$ are
	elements of $V$ and $c \in R$, define $(a_1, a_2) + (b_1, b_2) = (a_1 + b_1, a_2 b_2)$ and
	$c(a_1, a_2) = (ca_1, ca_2)$.
	\[
		\begin{split}
			d
		\end{split}
	\]
\end{homeworkProblem}

% Homework problem 5
\begin{homeworkProblem}
	Let $V = \{(a_1, a_2): a_1, a_2 \in R\}$. For $(a_1, a_2)$ and $(b_1, b_2)$ in $V$ and $c \in R$,
	defin $(a_1, a_2) + (b_1, b_2) = (a_1 + b_1, a_2 + b_2)$ and $c(a_1, a_2) = (ca_1, ca_2)$.
	$$ (a_1, a_2) + (b_1, b_2) = (a_1 + 2b_1, a_2 + 3b_2) $$
	Is $V$ a vector space over $R$ with these operations? Justify your answer. \\
	No, It does not hold for the commutative of addition property.
	\[
		\begin{split}
			x + y &=  y + x \\
			(a_1, a_2) + (b_1, b_2) &= (b_1, b_2) + (a_1, a_2) \\
			(a_1 + 2b_1, a_2 + 3b_2) &\neq (b_1 + 2a_1, b_2 + 3a_2) \\
		\end{split}
	\]
\end{homeworkProblem}

% Homework problem 6
\begin{homeworkProblem}
	Prove that $A + A^t$ is symmetric for any matrix $A$.
	\[
		\begin{split}
			(A + A^t)^t &= A^t + (A^t)^t \\
			&= A^t + A \\
			&= A + A^t
		\end{split}
	\]
\end{homeworkProblem}

% Homework problem 7
\begin{homeworkProblem}
	Let $W_1$, $W_3$, and $W_4$ be as in Excersise 8. Describe $W_1 \cap W_3$, $W_1 \cap W_4$, and $W_3 \cap W_4$,
	and observe that each is a subspace of $R^3$.
	\[
		\begin{split}
			d
		\end{split}
	\]
\end{homeworkProblem}

% Homework problem 8
\begin{homeworkProblem}
	Let $C^n(R)$ denote the set of all real-valued functions defined on the real line that have a continuous \textit{n}th derivative. Prove that
	$C^n(R)$ is a subspace of $\mathcal{F}(R,R)$.
	\[
		\begin{split}
			d
		\end{split}
	\]
\end{homeworkProblem}

\end{document}


