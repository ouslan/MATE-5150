\documentclass{article}

\usepackage{fancyhdr}
\usepackage{extramarks}
\usepackage{amsmath}
\usepackage{amsthm}
\usepackage{amsfonts}
\usepackage{tikz}
\usepackage[plain]{algorithm}
\usepackage{algpseudocode}
\usepackage[inline,shortlabels]{enumitem}
\usetikzlibrary{automata,positioning}

%
% Basic Document Settings
%

\topmargin=-0.45in
\evensidemargin=0in
\oddsidemargin=0in
\textwidth=6.5in
\textheight=9.0in
\headsep=0.25in

\linespread{1.1}

\pagestyle{fancy}
\lhead{\hmwkAuthorName}
\chead{\hmwkClass\ (\hmwkClassInstructor): \hmwkTitle}
\rhead{\firstxmark}
\lfoot{\lastxmark}
\cfoot{\thepage}

\renewcommand\headrulewidth{0.4pt}
\renewcommand\footrulewidth{0.4pt}

\setlength\parindent{0pt}

%
% Create Problem Sections
%

\newcommand{\enterProblemHeader}[1]{
	\nobreak\extramarks{}{Problem \arabic{#1} continued on next page\ldots}\nobreak{}
	\nobreak\extramarks{Problem \arabic{#1} (continued)}{Problem \arabic{#1} continued on next page\ldots}\nobreak{}
}

\newcommand{\exitProblemHeader}[1]{
	\nobreak\extramarks{Problem \arabic{#1} (continued)}{Problem \arabic{#1} continued on next page\ldots}\nobreak{}
	\stepcounter{#1}
	\nobreak\extramarks{Problem \arabic{#1}}{}\nobreak{}
}

\setcounter{secnumdepth}{0}
\newcounter{partCounter}
\newcounter{homeworkProblemCounter}
\setcounter{homeworkProblemCounter}{1}
\nobreak\extramarks{Problem \arabic{homeworkProblemCounter}}{}\nobreak{}

%
% Homework Problem Environment
%
% This environment takes an optional argument. When given, it will adjust the
% problem counter. This is useful for when the problems given for your
% assignment aren't sequential. See the last 3 problems of this template for an
% example.
%
\newenvironment{homeworkProblem}[1][-1]{
	\ifnum#1>0
		\setcounter{homeworkProblemCounter}{#1}
	\fi
	\section{Problem \arabic{homeworkProblemCounter}}
	\setcounter{partCounter}{1}
	\enterProblemHeader{homeworkProblemCounter}
}{
	\exitProblemHeader{homeworkProblemCounter}
}

%
% Homework Details
%   - Title
%   - Due date
%   - Class
%   - Section/Time
%   - Instructor
%   - Author
%

\newcommand{\hmwkTitle}{Asignacion\ \#1}
\newcommand{\hmwkDueDate}{Septiembre 5, 2024}
\newcommand{\hmwkClass}{MATE 5150}
\newcommand{\hmwkClassInstructor}{Dr. Pedro Vasquez}
\newcommand{\hmwkAuthorName}{\textbf{Alejandro Ouslan}}

%
% Title Page
%

\title{
	\vspace{2in}
	\textmd{\textbf{\hmwkClass:\ \hmwkTitle}}\\
	\normalsize\vspace{0.1in}\small{Due\ on\ \hmwkDueDate}\\
	\vspace{0.1in}\large{\textit{\hmwkClassInstructor}}
	\vspace{3in}
}

\author{\hmwkAuthorName}
\date{}

\renewcommand{\part}[1]{\textbf{\large Part \Alph{partCounter}}\stepcounter{partCounter}\\}

%
% Various Helper Commands
%

% Useful for algorithms
\newcommand{\alg}[1]{\textsc{\bfseries \footnotesize #1}}

% For derivatives
\newcommand{\deriv}[1]{\frac{\mathrm{d}}{\mathrm{d}x} (#1)}

% For partial derivatives
\newcommand{\pderiv}[2]{\frac{\partial}{\partial #1} (#2)}

% Integral dx
\newcommand{\dx}{\mathrm{d}x}
\newcommand{\R}{\mathbb{R}}
\newcommand{\Rm}{\mathbb{R}_{m \times n}}

% Alias for the Solution section header
\newcommand{\solution}{\textbf{\large Solution}}

% Probability commands: Expectation, Variance, Covariance, Bias
\newcommand{\E}{\mathrm{E}}
\newcommand{\Var}{\mathrm{Var}}
\newcommand{\Cov}{\mathrm{Cov}}
\newcommand{\Bias}{\mathrm{Bias}}

\begin{document}

\maketitle

\pagebreak

% Homework sec 1.2.3
\begin{homeworkProblem}
	Dadas las matrices
	\[
		A = \begin{bmatrix}
			2 & -\alpha & -1 \\ 3 & 0 & -1 \\ -1 & -1 & -2
		\end{bmatrix} \quad
		\text{y} \quad
		C = \begin{bmatrix}
			\beta & -2 & 1 \\ 2 & \beta & \beta \\ \alpha & 0 & \beta
		\end{bmatrix}
	\]
	\begin{itemize}
		\item Hallar $A + C$
		\item Para cual valor de $\alpha$ la matriz $A$ es simétrica?
		\item Para cuales valores de $\alpha$ y $\beta$ la matriz $A+B$ es antisimétrica?
		\item Para cuales valores de $\alpha$ y $\beta$ la matriz $A+B$ es triangular?
	\end{itemize}
\end{homeworkProblem}

% Homework sec 1.2.4
\begin{homeworkProblem}
	Pruebe que el elemento neutro para la suma en $\mathbb{R}_{m\times n}$ es único. Es decir, si
	$[0]$ y $[0]^T$ son tales que para toda matriz $A \in \mathbb{R}_{m \times n}$, $A + 0 = 0 + A =  A$ y
	$A + 0' = 0' + A = A$ entonces $0 = 0'$.
\end{homeworkProblem}

% Homework sec 1.2.5
\begin{homeworkProblem}
	Pruebe que para cada matriz $A \in \mathbb{R}_{m \times n}$, el inverso aditivo de $A$ es unico. Es decir,
	si $A +B = B + A = 0$ y $A + C = C + A = 0$ entonces $B = C$.
\end{homeworkProblem}

% Homework sec 1.2.6
\begin{homeworkProblem}
	Sea $A$ y $B$ matrixes de orden $m \times n$ y $\gamma$ un escalar. Pruebe que:
	\begin{enumerate}[(a)]
		\item $(A^T)^T = A$
		\item $(A + B)^T = A^T + B^T$
		\item $(\gamma A)^T = \gamma A^T$.
	\end{enumerate}
\end{homeworkProblem}

% Homework sec 1.2.7
\begin{homeworkProblem}
	Para cada condicion dada hallar una matriz $A$ que la satisfaga:
	\begin{enumerate}
		\item $A$ es a la vez triangular superior y triangular inferior.
		\item $A$ es a la vez simetrica y antisimetrica.
		\item $A^T=A$
		\item $A^T= -A$
	\end{enumerate}
\end{homeworkProblem}

% Homework sec 1.2.8
\begin{homeworkProblem}
	Sean $A$ y $B$ matrices cuadradas de orden $n$. Probar:
	\begin{enumerate}[(a)]
		\item Si $A$ es simetrica entonces, para todo escalar $\gamma$, $\gamma A$ es matriz simetrica.
		\item Si $A$ es antisimetrica entrocnes, para todo escalar $\gamma$, $\gamma A$ es antisimetrica.
		\item Si $A$ y $B$ son simetricas entonces $A + B$ es simetrica.
	\end{enumerate}
\end{homeworkProblem}

% Homework sec 1.2.9
\begin{homeworkProblem}
	Probar que para toda matriz cuadrada $A$,
	\begin{enumerate}[(a)]
		\item La matriz $A + A^T$ es simetrica y la matriz $A-A^T$ es antisimetrica.
		\item  La matriz $A$ puede expresarse de manera unica como la suma de una matriz simetrica y
		      una matriz antisimetrica.
		\item Sea $A = \begin{bmatrix}
				      2 & -1 \\ 3 & 5
			      \end{bmatrix}$. Descomponer la matriz $A$ como la suma de una matriz sismetrica y una matriz antisimetrica.
	\end{enumerate}
\end{homeworkProblem}

\end{document}
