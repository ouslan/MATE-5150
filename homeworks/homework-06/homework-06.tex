\documentclass{article}

\usepackage{fancyhdr}
\usepackage{extramarks}
\usepackage{amsmath}
\usepackage{amsthm}
\usepackage{amsfonts}
\usepackage{tikz}
\usepackage[plain]{algorithm}
\usepackage{algpseudocode}

\usetikzlibrary{automata,positioning}

%
% Basic Document Settings
%

\topmargin=-0.45in
\evensidemargin=0in
\oddsidemargin=0in
\textwidth=6.5in
\textheight=9.0in
\headsep=0.25in

\linespread{1.1}

\pagestyle{fancy}
\lhead{\hmwkAuthorName}
\chead{\hmwkClass\ (\hmwkClassInstructor): \hmwkTitle}
\rhead{\firstxmark}
\lfoot{\lastxmark}
\cfoot{\thepage}

\renewcommand\headrulewidth{0.4pt}
\renewcommand\footrulewidth{0.4pt}

\setlength\parindent{0pt}

%
% Create Problem Sections
%

\newcommand{\enterProblemHeader}[1]{
	\nobreak\extramarks{}{Problem \arabic{#1} continued on next page\ldots}\nobreak{}
	\nobreak\extramarks{Problem \arabic{#1} (continued)}{Problem \arabic{#1} continued on next page\ldots}\nobreak{}
}

\newcommand{\exitProblemHeader}[1]{
	\nobreak\extramarks{Problem \arabic{#1} (continued)}{Problem \arabic{#1} continued on next page\ldots}\nobreak{}
	\stepcounter{#1}
	\nobreak\extramarks{Problem \arabic{#1}}{}\nobreak{}
}

\setcounter{secnumdepth}{0}
\newcounter{partCounter}
\newcounter{homeworkProblemCounter}
\setcounter{homeworkProblemCounter}{1}
\nobreak\extramarks{Problem \arabic{homeworkProblemCounter}}{}\nobreak{}

%
% Homework Problem Environment
%
% This environment takes an optional argument. When given, it will adjust the
% problem counter. This is useful for when the problems given for your
% assignment aren't sequential. See the last 3 problems of this template for an
% example.
%
\newenvironment{homeworkProblem}[1][-1]{
	\ifnum#1>0
		\setcounter{homeworkProblemCounter}{#1}
	\fi
	\section{Problem \arabic{homeworkProblemCounter}}
	\setcounter{partCounter}{1}
	\enterProblemHeader{homeworkProblemCounter}
}{
	\exitProblemHeader{homeworkProblemCounter}
}

%
% Homework Details
%   - Title
%   - Due date
%   - Class
%   - Section/Time
%   - Instructor
%   - Author
%

\newcommand{\hmwkTitle}{Asignacion\ \#3}
\newcommand{\hmwkDueDate}{Septiembre 26, 2024}
\newcommand{\hmwkClass}{MATE 5150}
\newcommand{\hmwkClassInstructor}{Dr. Pedro Vasquez}
\newcommand{\hmwkAuthorName}{\textbf{Alejandro Ouslan}}

%
% Title Page
%

\title{
	\vspace{2in}
	\textmd{\textbf{\hmwkClass:\ \hmwkTitle}}\\
	\normalsize\vspace{0.1in}\small{Due\ on\ \hmwkDueDate}\\
	\vspace{0.1in}\large{\textit{\hmwkClassInstructor}}
	\vspace{3in}
}

\author{\hmwkAuthorName}
\date{}

\renewcommand{\part}[1]{\textbf{\large Part \Alph{partCounter}}\stepcounter{partCounter}\\}

%
% Various Helper Commands
%

% Useful for algorithms
\newcommand{\alg}[1]{\textsc{\bfseries \footnotesize #1}}

% For derivatives
\newcommand{\deriv}[1]{\frac{\mathrm{d}}{\mathrm{d}x} (#1)}

% For partial derivatives
\newcommand{\pderiv}[2]{\frac{\partial}{\partial #1} (#2)}

% Integral dx
\newcommand{\dx}{\mathrm{d}x}

% Alias for the Solution section header
\newcommand{\solution}{\textbf{\large Solution}}

% Probability commands: Expectation, Variance, Covariance, Bias
\newcommand{\E}{\mathrm{E}}
\newcommand{\Var}{\mathrm{Var}}
\newcommand{\Cov}{\mathrm{Cov}}
\newcommand{\Bias}{\mathrm{Bias}}

\begin{document}

\maketitle

\pagebreak

% Homework problem 1 sec 3.3 problem 2g
\begin{homeworkProblem}
	For each of the following homogeneous systems of linear equations, find the dimensions of and a basis for the solution space.
	\[
		\begin{array}{ll}
			 & x_1 + 2x_2 + x_3  + x_4 = 0           \\
			 & \quad \quad \quad x_2 - x_3 + x_3 = 0
		\end{array}
	\]
	\[
		\begin{split}
			\begin{bmatrix}
				1 & 2 & 1  & 1 & 0 \\
				0 & 1 & -1 & 1 & 0
			\end{bmatrix} \\
			x_2 &= x_3 - x_4 \\
			x_1 &= -x_3 - x_4 \\
			\begin{bmatrix}
				x_1 \\
				x_2 \\
				x_3 \\
				x_4 \\
			\end{bmatrix} &=
			\begin{bmatrix}
				-1 \\
				1  \\
				1  \\
				0  \\
			\end{bmatrix}x_3 +
			\begin{bmatrix}
				-1 \\
				-1 \\
				0  \\
				1  \\
			\end{bmatrix}x_4 \\
		\end{split}
	\]
	The rank of the matrix is 2, and dimension of the solution space is 2. A basis for the solution space is $\left\{(1, 1, 1, 0), (1, -1, 0, 1)\right\}$.
\end{homeworkProblem}

% Homework problem 2 sec 3.3 problem 3g
\begin{homeworkProblem}
	Using the results of Exercise 2, find all solutions to the following system of linear equations.
	\[
		\begin{array}{ll}
			x_1 + 2x_2 + x_3 + x_4 & = 0 \\
			x_2 - x_3 + x_3        & = 0
		\end{array}
	\]
	\[
		\begin{split}
			K = \left\{t_1\begin{bmatrix}
				-1 \\
				1  \\
				1  \\
				0  \\
			\end{bmatrix} + t_2
			\begin{bmatrix}
				-1 \\
				-1 \\
				0  \\
				1  \\
			\end{bmatrix} : t_1, t_2 \in \mathbb{R}\right\}
		\end{split}
	\]
\end{homeworkProblem}

% Homework problem 3 sec 3.3 problem 6
\begin{homeworkProblem}
	Let $T: \mathbb{R}^3 \to \mathbb{R}^2$ be defined by $T(a, b, c) = (a + b, 2a - c)$. Determine $T^{-1}(1, 11)$.
	\[
		\begin{split}
			T(a, b, c) &= (a + b, 2a - c) \\
			1 = a + b & \quad 11 = 2a - c \\
			b = 1 - a & \quad c = 2a - 11 \\
			(a, b, c) &= (a, 1 - a, 2a - 11) \\
			T^{-1}(1, 11) &= \left\{(a,1 - a, 2a - 11) : a \in \mathbb{R}\right\}
		\end{split}
	\]
\end{homeworkProblem}

% Homework problem 4 sec 3.3 problem 7d
\begin{homeworkProblem}
	Determine which of the following systems of liner equations has a solution.
	\[
		\begin{array}{ll}
			x_1 + x_2 + 3x_3 - x_3  & = 0 \\
			x_1 + x_2 + x_3 + x_4   & = 1 \\
			x_1 - 2x_2 + x_3 - x_4  & = 1 \\
			4x_1 + x_2 + 8x_3 - x_4 & = 0
		\end{array}
	\]
	\[
		\begin{split}
			A &= \begin{bmatrix}
				1 & 1  & 3 & -1 & 0 \\
				1 & 1  & 1 & 1  & 1 \\
				1 & -2 & 1 & -1 & 1 \\
				4 & 1  & 8 & -1 & 0
			\end{bmatrix} = \begin{bmatrix}
				1 & 0 & 0 & 0 & \frac{25}{6}  \\
				0 & 1 & 0 & 0 & \frac{4}{3}   \\
				0 & 0 & 1 & 0 & - \frac{5}{2} \\
				0 & 0 & 0 & 1 & -2            \\
			\end{bmatrix} \\
			s &= \begin{bmatrix}
				x_1 \\
				x_2 \\
				x_3 \\
				x_4 \\
			\end{bmatrix} = \begin{bmatrix}
				\frac{25}{6} \\
				\frac{4}{3}  \\
				-\frac{5}{2} \\
				-2           \\
			\end{bmatrix}
		\end{split}
	\]
\end{homeworkProblem}

% Homework problem 5 sec 3.3 problem 12
\begin{homeworkProblem}
	u	A certain economy consists of two sectors: goods and services. Suppose that 60\% of all goods and 30\% of all services are
	used in the production of goods. What proportion of the total economic output is used in the production of goods?

	\[
		\begin{split}
			P &= \frac{.6G + .3S}{G + S}  = \frac{.6G + .3S}{T_2} \\
			P &= \frac{.9G}{T_2}           = .45, \text{ where } G = 1, S = 1, T_2 = 2 \\
		\end{split}
	\]
\end{homeworkProblem}

% Homework problem 6 sec 3.4 problem 2g
\begin{homeworkProblem}
	Use Gaussian elimination to solve the following systems of linear equations.
	\[
		\begin{array}{ll}
			2x_1 - 2x_2 - x_3 + 6x_4 - 2x_5    & = 1 \\
			\quad x_1 - x_2 + x_3 + 2x_4 - x_5 & = 2 \\
			4x_1 - 4x_2 + 5x_3 + 7x_4 - x_5    & = 6
		\end{array}
	\]

	\[
		\begin{split}
			A = \begin{bmatrix}
				2 & -2 & -1 & 6 & -2 & 1 \\
				1 & -1 & 1  & 2 & -1 & 2 \\
				4 & -4 & 5  & 7 & -1 & 6 \\
			\end{bmatrix} \\
			\begin{bmatrix}
				1 & -1 & 0 & 0 & 23 & -23 \\
				0 & 0  & 1 & 0 & -6 & 7   \\
				0 & 0  & 0 & 1 & -9 & 9   \\
			\end{bmatrix} \\
			S = \begin{bmatrix}
				x_2 - 23x_5 - 23 \\
				x_2              \\
				6x_5 + 7         \\
				9x_5 + 9         \\
				x_5              \\
			\end{bmatrix} \\
		\end{split}
	\]
\end{homeworkProblem}

% Homework problem 7 sec 3.4 problem 5
\begin{homeworkProblem}
	Let the reduce row echolon form of $A$ be
	\[
		\begin{bmatrix}
			1 & 0 & 2  & 0 & -2 \\
			0 & 1 & -5 & 0 & -3 \\
			0 & 0 & 0  & 1 & 6  \\
		\end{bmatrix}
	\]
	Determine $A$ if the first, second , and forth columns of $A$ are
	\[
		\begin{bmatrix}
			1  \\
			-1 \\
			3  \\
		\end{bmatrix},
		\begin{bmatrix}
			0  \\
			-1 \\
			1  \\
		\end{bmatrix}, \text{ and }
		\begin{bmatrix}
			1  \\
			-2 \\
			0
		\end{bmatrix}
	\]
\end{homeworkProblem}

% Homework problem 8 sec 3.4 problem 9
\begin{homeworkProblem}
	Let $W$ be the subspace of $M_{2 \times 2} (\mathbb{R})$ consisting of the symmetric $2 \times 2$ matrices. Then set
	\[
		S = \left\{
		\begin{bmatrix}
			0  & -1 \\
			-1 & 1
		\end{bmatrix},
		\begin{bmatrix}
			1 & 2 \\
			2 & 3 \\
		\end{bmatrix},
		\begin{bmatrix}
			2 & 1 \\
			1 & 9
		\end{bmatrix},
		\begin{bmatrix}
			1  & -2 \\
			-2 & 4
		\end{bmatrix},
		\begin{bmatrix}
			-1 & -2 \\
			2  & -1
		\end{bmatrix}\right\}
	\]
	generates $W$. Find a subset of $S$ that is a basis for $W$.

	\[
		\begin{split}
			A_1 = \begin{bmatrix} 0 & -1 \\ -1 & 1 \end{bmatrix}, \\
			A_2 = \begin{bmatrix} 1 & 2 \\ 2 & 3 \end{bmatrix}, \\
			A_3 = \begin{bmatrix} 2 & 1 \\ 1 & 9 \end{bmatrix}, \\
			A_4 = \begin{bmatrix} 1 & -2 \\ -2 & 4 \end{bmatrix}, \\
			A_5 = \begin{bmatrix} -1 & -2 \\ 2 & -1 \end{bmatrix}. \\
			M = \begin{bmatrix}
				0  & 1 & 2 & 1  & -1 \\
				-1 & 2 & 1 & -2 & -2 \\
				1  & 3 & 9 & 4  & -1
			\end{bmatrix}. \\
			1. R_2 \leftarrow R_2 - 2R_1, \quad R_3 \leftarrow R_3 - 1R_1 \\
			\Rightarrow M \sim \begin{bmatrix}
				0  & 1 & 2  & 1  & -1 \\
				-1 & 0 & -3 & -4 & 1  \\
				1  & 3 & 9  & 4  & -1
			\end{bmatrix}. \\
			\Rightarrow \begin{bmatrix}
				-1 & 0 & -3 & -4 & 1  \\
				0  & 1 & 2  & 1  & -1 \\
				1  & 3 & 9  & 4  & -1
			\end{bmatrix}. \\
			3. R_1 \leftarrow -R_1 \text{ and then } R_3 \leftarrow R_3 + R_1 \\
			\Rightarrow \begin{bmatrix}
				1 & 0 & 3  & 4 & -1 \\
				0 & 1 & 2  & 1 & -1 \\
				0 & 3 & 12 & 8 & 0
			\end{bmatrix}. \\
			4. R_3 \leftarrow R_3 - 3R_2 \\
			\Rightarrow \begin{bmatrix}
				1 & 0 & 3 & 4 & -1 \\
				0 & 1 & 2 & 1 & -1 \\
				0 & 0 & 6 & 5 & 3
			\end{bmatrix}. \\
			\text{This shows there are 3 non-zero rows, hence } \text{dimension of span} = 3. \\
			\text{Thus, a basis for } W \text{ is } \left\{
			A_1, A_2, A_3 \right\}.
		\end{split}
	\]
\end{homeworkProblem}

% Homework problem 9 sec 3.4 problem 13
\begin{homeworkProblem}
	Let $V$ be a in Exercise 12
	\[
		\begin{array}{ll}
			x_1 - x_2 + 2x_4 - 3x_5 + x_6         & = 0 \\
			2x_1 - x_2 - x_3 + 3x_4 - 4x_5 + 4x_6 & = 0 \\
		\end{array}
	\]
	\begin{enumerate}
		\item Show that $S = \left\{(0,1,0,1,1,0), (0,2,1,1,0,0)\right\}$ is a linerly independent subset of $V$.
		\item Extend $S$ to a basis for $V$.
	\end{enumerate}
\end{homeworkProblem}

\end{document}
