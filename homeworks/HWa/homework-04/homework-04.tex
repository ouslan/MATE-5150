
\documentclass{article}

\usepackage{fancyhdr}
\usepackage{extramarks}
\usepackage{amsmath}
\usepackage{amsthm}
\usepackage{amsfonts}
\usepackage{tikz}
\usepackage[plain]{algorithm}
\usepackage{algpseudocode}

\usetikzlibrary{automata,positioning}

%
% Basic Document Settings
%

\topmargin=-0.45in
\evensidemargin=0in
\oddsidemargin=0in
\textwidth=6.5in
\textheight=9.0in
\headsep=0.25in

\linespread{1.1}

\pagestyle{fancy}
\lhead{\hmwkAuthorName}
\chead{\hmwkClass\ (\hmwkClassInstructor): \hmwkTitle}
\rhead{\firstxmark}
\lfoot{\lastxmark}
\cfoot{\thepage}

\renewcommand\headrulewidth{0.4pt}
\renewcommand\footrulewidth{0.4pt}

\setlength\parindent{0pt}

%
% Create Problem Sections
%

\newcommand{\enterProblemHeader}[1]{
	\nobreak\extramarks{}{Problem \arabic{#1} continued on next page\ldots}\nobreak{}
	\nobreak\extramarks{Problem \arabic{#1} (continued)}{Problem \arabic{#1} continued on next page\ldots}\nobreak{}
}

\newcommand{\exitProblemHeader}[1]{
	\nobreak\extramarks{Problem \arabic{#1} (continued)}{Problem \arabic{#1} continued on next page\ldots}\nobreak{}
	\stepcounter{#1}
	\nobreak\extramarks{Problem \arabic{#1}}{}\nobreak{}
}

\setcounter{secnumdepth}{0}
\newcounter{partCounter}
\newcounter{homeworkProblemCounter}
\setcounter{homeworkProblemCounter}{1}
\nobreak\extramarks{Problem \arabic{homeworkProblemCounter}}{}\nobreak{}

%
% Homework Problem Environment
%
% This environment takes an optional argument. When given, it will adjust the
% problem counter. This is useful for when the problems given for your
% assignment aren't sequential. See the last 3 problems of this template for an
% example.
%
\newenvironment{homeworkProblem}[1][-1]{
	\ifnum#1>0
		\setcounter{homeworkProblemCounter}{#1}
	\fi
	\section{Problem \arabic{homeworkProblemCounter}}
	\setcounter{partCounter}{1}
	\enterProblemHeader{homeworkProblemCounter}
}{
	\exitProblemHeader{homeworkProblemCounter}
}

%
% Homework Details
%   - Title
%   - Due date
%   - Class
%   - Section/Time
%   - Instructor
%   - Author
%

\newcommand{\hmwkTitle}{Asignacion\ \#3}
\newcommand{\hmwkDueDate}{Septiembre 26, 2024}
\newcommand{\hmwkClass}{MATE 5150}
\newcommand{\hmwkClassInstructor}{Dr. Pedro Vasquez}
\newcommand{\hmwkAuthorName}{\textbf{Alejandro Ouslan}}

%
% Title Page
%

\title{
	\vspace{2in}
	\textmd{\textbf{\hmwkClass:\ \hmwkTitle}}\\
	\normalsize\vspace{0.1in}\small{Due\ on\ \hmwkDueDate}\\
	\vspace{0.1in}\large{\textit{\hmwkClassInstructor}}
	\vspace{3in}
}

\author{\hmwkAuthorName}
\date{}

\renewcommand{\part}[1]{\textbf{\large Part \Alph{partCounter}}\stepcounter{partCounter}\\}

%
% Various Helper Commands
%

% Useful for algorithms
\newcommand{\alg}[1]{\textsc{\bfseries \footnotesize #1}}

% For derivatives
\newcommand{\deriv}[1]{\frac{\mathrm{d}}{\mathrm{d}x} (#1)}

% For partial derivatives
\newcommand{\pderiv}[2]{\frac{\partial}{\partial #1} (#2)}

% Integral dx
\newcommand{\dx}{\mathrm{d}x}

% Alias for the Solution section header
\newcommand{\solution}{\textbf{\large Solution}}

% Probability commands: Expectation, Variance, Covariance, Bias
\newcommand{\E}{\mathrm{E}}
\newcommand{\Var}{\mathrm{Var}}
\newcommand{\Cov}{\mathrm{Cov}}
\newcommand{\Bias}{\mathrm{Bias}}

\begin{document}

\maketitle

\pagebreak

% Homework problem 1 sec 2.4, prob 2b
\begin{homeworkProblem}
	Determine whether $T$ is invertible and justify your answer.
	$$T: \mathbb{R}^2 \rightarrow \mathbb{R}^3, T(a_1, a_2) = (3a_1 - a_2, a_2,4a_1)$$
	Since the $Rank(\mathbb{R}^3) \neq Dim(\mathbb{R}^2)$, then $T$ is not invertible.
\end{homeworkProblem}

% Homework problem 2 sec 2.4, prob 2f
\begin{homeworkProblem}
	Determine whether $T$ is invertible and justify your answer.
	\[
		T: M_{2 \times 2}(\mathbb{R}) \rightarrow M_{2 \times 2}(\mathbb{R}),
		T\left(\begin{bmatrix} a & b \\ c & d \end{bmatrix}\right) = \begin{bmatrix} a + b & a \\ c & c + d \end{bmatrix}
	\]
	\[
		\begin{split}
			\text{Check if } T \text{ is linear:} \\
			T(A + B) &= T\left(\begin{bmatrix} a_1 + a_2 & b_1 + b_2 \\ c_1 + c_2 & d_1 + d_2 \end{bmatrix}\right) \\
			&= \begin{bmatrix} (a_1 + a_2) + (b_1 + b_2) & (a_1 + a_2) \\ (c_1 + c_2) & (c_1 + c_2) + (d_1 + d_2) \end{bmatrix} \\
			&= \begin{bmatrix} a_1 + b_1 + a_2 + b_2 & a_1 + a_2 \\ c_1 + c_2 & c_1 + c_2 + d_1 + d_2 \end{bmatrix} \\
			T(A) + T(B) &= \begin{bmatrix} (a_1 + b_1) + (a_2 + b_2) & a_1 + a_2 \\ c_1 + c_2 & (c_1 + d_1) + (c_2 + d_2) \end{bmatrix} \\
			&= \begin{bmatrix} a_1 + b_1 + a_2 + b_2 & a_1 + a_2 \\ c_1 + c_2 & c_1 + c_2 + d_1 + d_2 \end{bmatrix} \\
			&\Rightarrow T(A + B) = T(A) + T(B). \\
			cT(A): &= c \begin{bmatrix} a_1 + b_1 & a_1 \\ c_1 & c_1 + d_1 \end{bmatrix} \\
			&= \begin{bmatrix} c(a_1 + b_1) & ca_1 \\ cc_1 & c(c_1 + d_1) \end{bmatrix} \\
			T(cA) &= T\left(\begin{bmatrix} ca_1 & cb_1 \\ cc_1 & cd_1 \end{bmatrix}\right) \\
			&= \begin{bmatrix} ca_1 + cb_1 & ca_1 \\ cc_1 & cc_1 + cd_1 \end{bmatrix} \\
			&\Rightarrow T(cA) = cT(A). \\
			\text{Find the kernel of } T: \\
			\ker(T) &= \{ X \in M_{2 \times 2}(\mathbb{R}) : T(X) = 0 \} \\
			T\left(\begin{bmatrix} a & b \\ c & d \end{bmatrix}\right) &= \begin{bmatrix} 0 & 0 \\ 0 & 0 \end{bmatrix} \\
			&\Rightarrow a + b = 0, a = 0, c = 0, c + d = 0 \\
			&\Rightarrow a = b = c = d = 0 \\
			&\Rightarrow \ker(T) = \{ 0 \}. \\
			\text{Determine the range of } T: \\
			T\left(\begin{bmatrix} a & b \\ c & d \end{bmatrix}\right) &= \begin{bmatrix} a + b & a \\ c & c + d \end{bmatrix}. \\
			\text{Let } \begin{bmatrix} x_1 & x_2 \\ y_1 & y_2 \end{bmatrix} \text{ be in the range.} \\
			x_1 &= a + b, x_2 = a, y_1 = c, y_2 = c + d. \\
			&\Rightarrow a = x_2, b = x_1 - x_2, c = y_1, d = y_2 - y_1. \\
		\end{split}
	\]
	Therefore, $T$ is invertible.
\end{homeworkProblem}


% Homework problem 3 sec 2.4, prob 3d

\begin{homeworkProblem}
	Is the following pairs of vector spaces are isomorphic? Justify your answer.
	$$V = \{A \in M_{2 \times 2}(\mathbb{R}): \text{tr}(A) = 0\} \text{ and } \mathbb{R}^4$$
	\[
		\begin{split}
			\text{Dimension of } \mathbb{R}^4: & \quad \text{The vector space } \mathbb{R}^4 \text{ has dimension } 4. \\[6pt]
			\text{Dimension of } V: & \quad \text{The space } M_{2 \times 2}(\mathbb{R}) \text{ has dimension } 4. \\
			& \quad \text{The condition } \text{tr}(A) = 0 \text{ implies } a + d = 0. \\
			& \quad \text{Thus, a general element in } V \text{ can be written as:} \\
			& \quad A = \begin{pmatrix} a & b \\ c & -a \end{pmatrix}. \\
			& \quad \text{The parameters } a, b, c \text{ can be chosen freely, leading to a dimension of } 3. \\[6pt]
			\text{Comparing dimensions:} & \quad \text{Dimension of } V \text{ is } 3. \\
			& \quad \text{Dimension of } \mathbb{R}^4 \text{ is } 4. \\
			& \quad \text{Since the dimensions are different, } V \text{ cannot be isomorphic to } \mathbb{R}^4. \\[6pt]
			\text{Conclusion:} & \quad \text{The vector spaces } V \text{ and } \mathbb{R}^4 \text{ are not isomorphic.}
		\end{split}
	\]
\end{homeworkProblem}

% Homework problem 4 sec 2.4, prob 14

\begin{homeworkProblem}
	Let
	\[
		V = \left\{ \begin{bmatrix} a & a+b \\ 0 & c \end{bmatrix} : a, b, c \in F \right\}
	\]
	Construct an isomorphism from \(V\) to \(F^3\).
	\[
		\begin{split}
			T: V & \to F^3 \\
			T\left(\begin{bmatrix} a & a+b \\ 0 & c \end{bmatrix}\right) & = (a, b, c) \\
			\text{To show } T \text{ is linear:} & \\
			\text{Additivity:} & \\
			T\left(A + B\right) & = T\left(\begin{bmatrix} a_1 + a_2 & (a_1+b_1) + (a_2+b_2) \\ 0 & c_1 + c_2 \end{bmatrix}\right) = (a_1 + a_2, b_1 + b_2, c_1 + c_2) \\
			& = T(A) + T(B) \\
			\text{Scalar multiplication:} & \\
			T(kA) & = T\left(\begin{bmatrix} ka & k(a+b) \\ 0 & kc \end{bmatrix}\right) = (ka, kb, kc) = k T(A) \\
			\text{Injectivity:} & \quad T(A) = T(B) \implies A = B \\
			\text{Surjectivity:} & \quad \forall (a, b, c) \in F^3, \exists \begin{bmatrix} a & a+b \\ 0 & c \end{bmatrix} \in V \\
			\text{Conclusion:} & \quad T \text{ is an isomorphism.}
		\end{split}
	\]
\end{homeworkProblem}


% Homework problem 5 sec 2.5, prob 2b
\begin{homeworkProblem}
	For each of the following pairs of ordered bases $\beta$ and $\beta'$ for $\mathbb{R}^2$,
	find the change of coordinates matrix that changes $\beta$-coordinates into $\beta$-coordinate
	$$ \beta = \{(-1,3), (2,-1)\} \text{ and } \beta' = \{(0,10),(5,0)\}$$
	\[
		\begin{split}
			\text{Let } \beta &= \{ \mathbf{v_1} = (-1, 3), \mathbf{v_2} = (2, -1) \} \\
			\text{Let } \beta' &= \{ \mathbf{u_1} = (0, 10), \mathbf{u_2} = (5, 0) \} \\
			\text{Solve } \mathbf{v_1} &= a_1 \mathbf{u_1} + b_1 \mathbf{u_2} \\
			(-1, 3) &= a_1(0, 10) + b_1(5, 0) \\
			5b_1 &= -1 \quad \Rightarrow \quad b_1 = -\frac{1}{5} \\
			10a_1 &= 3 \quad \Rightarrow \quad a_1 = \frac{3}{10} \\
			\text{Solve } \mathbf{v_2} &= a_2 \mathbf{u_1} + b_2 \mathbf{u_2} \\
			(2, -1) &= a_2(0, 10) + b_2(5, 0) \\
			5b_2 &= 2 \quad \Rightarrow \quad b_2 = \frac{2}{5} \\
			10a_2 &= -1 \quad \Rightarrow \quad a_2 = -\frac{1}{10} \\
			\text{Change of coordinates matrix } P &= \begin{pmatrix}
				a_1 & a_2 \\
				b_1 & b_2
			\end{pmatrix} = \begin{pmatrix}
				\frac{3}{10} & -\frac{1}{10} \\
				-\frac{1}{5} & \frac{2}{5}
			\end{pmatrix}
		\end{split}
	\]
\end{homeworkProblem}

% Homework problem 6 sec 2.5, prob 3b
\begin{homeworkProblem}
	For each of the following pairs of ordered bases $\beta$ and $\beta'$ for $P_2(\mathbb{R})$,
	find the change of coordinate matrix that changes $\beta'$-coordinates into $\beta$-coordinates.
	$$\beta = \{1,x,x^2\} \text{ and } \beta' = \{a_2x^2 + a_1x + a_0, b_2x^2 + b_1x + b_0, c_2x^2 + c_1x + c_0\}$$
	\[
		\begin{split}
			\text{Let } \beta' = \{ \mathbf{u_1}, \mathbf{u_2}, \mathbf{u_3} \} \\
			\text{Express } \mathbf{u_1} = a_2 x^2 + a_1 x + a_0 \text{ in terms of } \beta: \\
			\mathbf{u_1} = a_2 x^2 + a_1 x + a_0 = c_1(1) + c_2(x) + c_3(x^2) \\
			\text{Coefficients: } \begin{cases}
				c_1 = a_0 \\
				c_2 = a_1 \\
				c_3 = a_2
			\end{cases} \\
			\text{Express } \mathbf{u_2} = b_2 x^2 + b_1 x + b_0 \text{ in terms of } \beta: \\
			\mathbf{u_2} = b_2 x^2 + b_1 x + b_0 = c_1(1) + c_2(x) + c_3(x^2) \\
			\text{Coefficients: } \begin{cases}
				c_1 = b_0 \\
				c_2 = b_1 \\
				c_3 = b_2
			\end{cases} \\
			\text{Express } \mathbf{u_3} &= c_2 x^2 + c_1 x + c_0 \text{ in terms of } \beta: \\
			\mathbf{u_3} = c_2 x^2 + c_1 x + c_0 = c_1(1) + c_2(x) + c_3(x^2) \\
			\text{Coefficients: } \begin{cases}
				c_1 = c_0 \\
				c_2 = c_1 \\
				c_3 = c_2
			\end{cases} \\
			\text{Change of coordinates matrix } P &= \begin{pmatrix}
				a_0 & b_0 & c_0 \\
				a_1 & b_1 & c_1 \\
				a_2 & b_2 & c_2
			\end{pmatrix}
		\end{split}
	\]
\end{homeworkProblem}

% Homework problem 7 sec 2.5, prob 4
\begin{homeworkProblem}
	Let $T$ be the linear operator on $\mathbb{R}^2$ defined by
	$$T\begin{pmatrix} a \\ b \end{pmatrix} = \begin{pmatrix} 2a + b \\ a - 3b \end{pmatrix}$$
	let $\beta$ be the standard ordered basis for $\mathbb{R}^2$, and let
	$$\beta' = \left\{ \begin{pmatrix} 1 \\ 1 \end{pmatrix}, \begin{pmatrix} 1 \\ 2 \end{pmatrix} \right\}$$
	Use Theorem 2.23 and the fact that
	$$\begin{pmatrix} 1 & 1 \\ 1 & 2 \end{pmatrix}^{-1} = \begin{pmatrix} 2 & -1 \\ -1 & 1 \end{pmatrix}$$
	to find $[T]_{\beta'}$.
	\[
		\begin{split}
			[T]_{\beta} = \begin{pmatrix}
				2 & 1  \\
				1 & -3
			\end{pmatrix} \\
			\text{Let } P = \begin{pmatrix} 1 & 1 \\ 1 & 2 \end{pmatrix}, \text{ then } P^{-1} = \begin{pmatrix} 2 & -1 \\ -1 & 1 \end{pmatrix} \\
			\text{Use the formula } [T]_{\beta'} = P^{-1}[T]_{\beta}P: \\
			[T]_{\beta'} &= \begin{pmatrix} 2 & -1 \\ -1 & 1 \end{pmatrix} \begin{pmatrix}
				2 & 1  \\
				1 & -3
			\end{pmatrix} \begin{pmatrix} 1 & 1 \\ 1 & 2 \end{pmatrix} \\
			\text{Calculate } [T]_{\beta}P: \\
			[T]_{\beta}P &= \begin{pmatrix}
				2 & 1  \\
				1 & -3
			\end{pmatrix} \begin{pmatrix} 1 & 1 \\ 1 & 2 \end{pmatrix} = \begin{pmatrix}
				3  & 4  \\
				-2 & -5
			\end{pmatrix} \\
			\text{Calculate } P^{-1}[T]_{\beta}P: \\
			[T]_{\beta'} = \begin{pmatrix} 2 & -1 \\ -1 & 1 \end{pmatrix} \begin{pmatrix}
				3  & 4  \\
				-2 & -5
			\end{pmatrix} \\
			\begin{pmatrix}
				2 \cdot 3 + (-1)(-2)      & 2 \cdot 4 + (-1)(-5)      \\
				-1 \cdot 3 + 1 \cdot (-2) & -1 \cdot 4 + 1 \cdot (-5)
			\end{pmatrix} = \begin{pmatrix}
				8  & 13 \\
				-5 & -9
			\end{pmatrix} \\
			\text{Thus, } [T]_{\beta'} = \begin{pmatrix} 8 & 13 \\ -5 & -9 \end{pmatrix}.
		\end{split}
	\]
\end{homeworkProblem}

% Homework problem 8 sec 2.5, prob 6b
\begin{homeworkProblem}
	For each matrix $A$ and ordered basis $\beta$, find $[L_A]_{\beta}$. Also, find an
	invertible matrix $Q$ such that $[L_A]_{\beta} = Q^{-1}AQ$.
	$$A = \begin{pmatrix} 1 & 2 \\ 2 & 1 \end{pmatrix} \text{ and } \beta = \left\{ \begin{pmatrix} 1 \\ 1 \end{pmatrix}, \begin{pmatrix} 1 \\ -1 \end{pmatrix} \right\}$$
	\[
		\begin{split}
			\text{Let } \mathbf{v_1} &= \begin{pmatrix} 1 \\ 1 \end{pmatrix}, \quad \mathbf{v_2} = \begin{pmatrix} 1 \\ -1 \end{pmatrix}. \\
			\text{Compute } A\mathbf{v_1} \text{ and } A\mathbf{v_2}: \\
			A\mathbf{v_1} = A \begin{pmatrix} 1 \\ 1 \end{pmatrix} = \begin{pmatrix} 1 & 2 \\ 2 & 1 \end{pmatrix} \begin{pmatrix} 1 \\ 1 \end{pmatrix} = \begin{pmatrix} 3 \\ 3 \end{pmatrix} = 3\mathbf{v_1}, \\
			A\mathbf{v_2} = A \begin{pmatrix} 1 \\ -1 \end{pmatrix} = \begin{pmatrix} 1 & 2 \\ 2 & 1 \end{pmatrix} \begin{pmatrix} 1 \\ -1 \end{pmatrix} = \begin{pmatrix} 1 - 2 \\ 2 - 1 \end{pmatrix} = \begin{pmatrix} -1 \\ 1 \end{pmatrix} = -\mathbf{v_2}. \\
			\text{Thus, } [L_A]_{\beta} = \begin{pmatrix} 3 & 0 \\ 0 & -1 \end{pmatrix}. \\
			\text{To find } Q, \text{ use } Q = \begin{pmatrix} 1 & 1 \\ 1 & -1 \end{pmatrix}. \\
			\text{Now, compute } Q^{-1}: \\
			Q^{-1} = \frac{1}{\det(Q)} \text{ adj}(Q) = \frac{1}{2} \begin{pmatrix} 1 & 1 \\ 1 & -1 \end{pmatrix} = \begin{pmatrix} \frac{1}{2} & \frac{1}{2} \\ \frac{1}{2} & -\frac{1}{2} \end{pmatrix}. \\
			\text{Confirm that } [L_A]_{\beta} = Q^{-1}AQ: \\
			Q^{-1}AQ = \begin{pmatrix} \frac{1}{2} & \frac{1}{2} \\ \frac{1}{2} & -\frac{1}{2} \end{pmatrix} \begin{pmatrix} 1 & 2 \\ 2 & 1 \end{pmatrix} \begin{pmatrix} 1 & 1 \\ 1 & -1 \end{pmatrix}. \\
			\text{Compute } AQ: \\
			AQ = \begin{pmatrix} 1 & 2 \\ 2 & 1 \end{pmatrix} \begin{pmatrix} 1 & 1 \\ 1 & -1 \end{pmatrix} = \begin{pmatrix} 3 & -1 \\ 1 & 1 \end{pmatrix}. \\
			\text{Then compute } Q^{-1}AQ: \\
			= \begin{pmatrix} \frac{1}{2} & \frac{1}{2} \\ \frac{1}{2} & -\frac{1}{2} \end{pmatrix} \begin{pmatrix} 3 & -1 \\ 1 & 1 \end{pmatrix} = \begin{pmatrix} 1 & 1 \\ 1 & -1 \end{pmatrix} = \begin{pmatrix} 3 & 0 \\ 0 & -1 \end{pmatrix}. \\
			\text{Thus, } [L_A]_{\beta} = \begin{pmatrix} 3 & 0 \\ 0 & -1 \end{pmatrix} \text{ and } Q = \begin{pmatrix} 1 & 1 \\ 1 & -1 \end{pmatrix}.
		\end{split}
	\]
\end{homeworkProblem}


\end{document}
