\documentclass{article}

\usepackage{fancyhdr}
\usepackage{extramarks}
\usepackage{amsmath}
\usepackage{amsthm}
\usepackage{amsfonts}
\usepackage{tikz}
\usepackage[plain]{algorithm}
\usepackage{algpseudocode}

\usetikzlibrary{automata,positioning}

%
% Basic Document Settings
%

\topmargin=-0.45in
\evensidemargin=0in
\oddsidemargin=0in
\textwidth=6.5in
\textheight=9.0in
\headsep=0.25in

\linespread{1.1}

\pagestyle{fancy}
\lhead{\hmwkAuthorName}
\chead{\hmwkClass\ (\hmwkClassInstructor): \hmwkTitle}
\rhead{\firstxmark}
\lfoot{\lastxmark}
\cfoot{\thepage}

\renewcommand\headrulewidth{0.4pt}
\renewcommand\footrulewidth{0.4pt}

\setlength\parindent{0pt}

%
% Create Problem Sections
%

\newcommand{\enterProblemHeader}[1]{
	\nobreak\extramarks{}{Problem \arabic{#1} continued on next page\ldots}\nobreak{}
	\nobreak\extramarks{Problem \arabic{#1} (continued)}{Problem \arabic{#1} continued on next page\ldots}\nobreak{}
}

\newcommand{\exitProblemHeader}[1]{
	\nobreak\extramarks{Problem \arabic{#1} (continued)}{Problem \arabic{#1} continued on next page\ldots}\nobreak{}
	\stepcounter{#1}
	\nobreak\extramarks{Problem \arabic{#1}}{}\nobreak{}
}

\setcounter{secnumdepth}{0}
\newcounter{partCounter}
\newcounter{homeworkProblemCounter}
\setcounter{homeworkProblemCounter}{1}
\nobreak\extramarks{Problem \arabic{homeworkProblemCounter}}{}\nobreak{}

%
% Homework Problem Environment
%
% This environment takes an optional argument. When given, it will adjust the
% problem counter. This is useful for when the problems given for your
% assignment aren't sequential. See the last 3 problems of this template for an
% example.
%
\newenvironment{homeworkProblem}[1][-1]{
	\ifnum#1>0
		\setcounter{homeworkProblemCounter}{#1}
	\fi
	\section{Problem \arabic{homeworkProblemCounter}}
	\setcounter{partCounter}{1}
	\enterProblemHeader{homeworkProblemCounter}
}{
	\exitProblemHeader{homeworkProblemCounter}
}

%
% Homework Details
%   - Title
%   - Due date
%   - Class
%   - Section/Time
%   - Instructor
%   - Author
%

\newcommand{\hmwkTitle}{Asignacion\ \#7}
\newcommand{\hmwkDueDate}{Noviembre 14, 2024}
\newcommand{\hmwkClass}{MATE 5150}
\newcommand{\hmwkClassInstructor}{Dr. Pedro Vasquez}
\newcommand{\hmwkAuthorName}{\textbf{Alejandro Ouslan}}

%
% Title Page
%

\title{
	\vspace{2in}
	\textmd{\textbf{\hmwkClass:\ \hmwkTitle}}\\
	\normalsize\vspace{0.1in}\small{Due\ on\ \hmwkDueDate}\\
	\vspace{0.1in}\large{\textit{\hmwkClassInstructor}}
	\vspace{3in}
}

\author{\hmwkAuthorName}
\date{}

\renewcommand{\part}[1]{\textbf{\large Part \Alph{partCounter}}\stepcounter{partCounter}\\}

%
% Various Helper Commands
%

% Useful for algorithms
\newcommand{\alg}[1]{\textsc{\bfseries \footnotesize #1}}

% For derivatives
\newcommand{\deriv}[1]{\frac{\mathrm{d}}{\mathrm{d}x} (#1)}

% For partial derivatives
\newcommand{\pderiv}[2]{\frac{\partial}{\partial #1} (#2)}

% Integral dx
\newcommand{\dx}{\mathrm{d}x}

% Alias for the Solution section header
\newcommand{\solution}{\textbf{\large Solution}}

% Probability commands: Expectation, Variance, Covariance, Bias
\newcommand{\E}{\mathrm{E}}
\newcommand{\Var}{\mathrm{Var}}
\newcommand{\Cov}{\mathrm{Cov}}
\newcommand{\Bias}{\mathrm{Bias}}

\begin{document}

\maketitle

\pagebreak

% Homework problem 1 sec 5.1 problem 3c
\begin{homeworkProblem}
	For each of the following linear operators $T$ on a vector space $V$ and ordered basis $\beta$, compute $[T]_{\beta}$ and determine
	whether $\beta$ is a basis consisting of eigenvectors of $T$.
	\[
		V = \mathbb{R}^3, T \begin{pmatrix} a \\ b \\ c\end{pmatrix} =
		\begin{pmatrix} 3a + 2b - 2c \\ -4a - 3b + 2c \\ -c \end{pmatrix}, \text{ and }
		\beta = \left\{ \begin{pmatrix} 0 \\ 1 \\ 1 \end{pmatrix}, \begin{pmatrix} 1 \\ -1 \\ 0 \end{pmatrix}, \begin{pmatrix} 1 \\ 0 \\ 2 \end{pmatrix} \right\}
	\]
	Compute $[T]_{\beta}$:
	\[
		\begin{split}
			\begin{pmatrix} 3(0) + 2(1) - 2(1) \\ -4(0) - 3(1) + 2(1) \\ -(1) \end{pmatrix}                                                       & = \begin{pmatrix} 0 \\ -1 \\ -1 \end{pmatrix}                         \\
			a\begin{pmatrix} 0 \\ 1 \\ 1 \end{pmatrix} + b\begin{pmatrix} 1 \\ -1 \\ 0 \end{pmatrix} + c\begin{pmatrix} 1 \\ 0 \\ 2 \end{pmatrix} & = \begin{pmatrix} 0 \\ -1 \\ -1 \end{pmatrix}                         \\
			v_1                                                                                                                                   & = \begin{pmatrix} -1 \\ 0 \\ 1 \end{pmatrix}                          \\
			v_2                                                                                                                                   & = \begin{pmatrix} 0 \\ 1 \\ 0 \end{pmatrix}                           \\
			v_3                                                                                                                                   & = \begin{pmatrix} 0 \\ 0 \\ -1 \end{pmatrix}                          \\
			[T]_\beta                                                                                                                             & = \begin{pmatrix} -1 & 0 & 0 \\ 0 & 1 & 0 \\ 0 & 0 & -1 \end{pmatrix} \\
		\end{split}
	\]
	Finding the eigenvalues of $T$:
	\[
		\begin{split}
			\text{det}([T]_\beta - \lambda I) & = \text{det}\left(\begin{pmatrix} -1 - \lambda & 0 & 0 \\ 0 & 1 - \lambda & 0 \\ 0 & 0 & -1 - \lambda \end{pmatrix}\right) \\
			                                  & = (1-\lambda)(1+\lambda)^2 = 0                                                                                             \\
			\lambda_1, \lambda_2              & = 1, -1
		\end{split}
	\]
	Calculating the eigenvectors for $\lambda_1 = 1$:
	\[
		\begin{split}
			\begin{pmatrix} -1 - 1 & 0 & 0 \\ 0 & 1 - 1 & 0 \\ 0 & 0 & -1 - 1 \end{pmatrix}\begin{pmatrix} x \\ y \\ z \end{pmatrix} & = \begin{pmatrix} 0 \\ 0 \\ 0 \end{pmatrix} \\
			\begin{pmatrix} -2 & 0 & 0 \\ 0 & 0 & 0 \\ 0 & 0 & -2 \end{pmatrix}\begin{pmatrix} x \\ y \\ z \end{pmatrix}             & = \begin{pmatrix} 0 \\ 0 \\ 0 \end{pmatrix} \\
			v_1                                                                                                                      & = \begin{pmatrix} 0 \\ 1 \\ 0 \end{pmatrix}
		\end{split}
	\]
	Calculating the eigenvectors for $\lambda_2 = -1$:
	\[
		\begin{split}
			\begin{pmatrix} -1 + 1 & 0 & 0 \\ 0 & 1 + 1 & 0 \\ 0 & 0 & -1 + 1 \end{pmatrix}\begin{pmatrix} x \\ y \\ z \end{pmatrix} & = \begin{pmatrix} 0 \\ 0 \\ 0 \end{pmatrix} \\
			\begin{pmatrix} 0 & 0 & 0 \\ 0 & 2 & 0 \\ 0 & 0 & 0 \end{pmatrix}\begin{pmatrix} x \\ y \\ z \end{pmatrix}               & = \begin{pmatrix} 0 \\ 0 \\ 0 \end{pmatrix} \\
			v_2                                                                                                                      & = \begin{pmatrix} 1 \\ 0 \\ 0 \end{pmatrix} \\
			v_3                                                                                                                      & = \begin{pmatrix} 0 \\ 0 \\ 1 \end{pmatrix}
		\end{split}
	\]

\end{homeworkProblem}

% Homework problem 2 sec 5.1 problem 4b
\begin{homeworkProblem}
	For each of the following matrices $A \in M_{n \times n}(F)$,
	\begin{enumerate}
		\item Determine all the eigenvalues of $A$.
		\item For each eigenvalue $\lambda$ of $A$, find the set of eigenvecotrs corresponding to $\lambda$.
		\item If possible, find a basis for $F^n$ consisting of eigenvectors of $A$.
		\item If successful in finding such a basis, determine an invertible matrix $Q$ and a diagonal matrix $D$ such that $Q^{-1}AQ = D$.
	\end{enumerate}
	\[
		A = \begin{pmatrix} 0 & -2 & -3 \\ -1 & 1 & -1 \\ 2 & 2 & 5 \end{pmatrix} \text{ for } F = \mathbb{R}
	\]
	Determine all the eigenvalues of $A$:
	\[
		\begin{split}
			\text{det}(A - \lambda I)       & = \text{det}\left(\begin{pmatrix} 0 - \lambda & -2 & -3 \\ -1 & 1 - \lambda & -1 \\ 2 & 2 & 5 - \lambda \end{pmatrix}\right) \\
			                                & = (3-\lambda)(2-\lambda)(1-\lambda) = 0                                                                                      \\
			\lambda_1, \lambda_2, \lambda_3 & = 3, 2, 1
		\end{split}
	\]
	Calculating the eigenvectors for $\lambda_1 = 3$:
	\[
		\begin{split}
			\begin{pmatrix} 0 - 3 & -2 & -3 \\ -1 & 1 - 3 & -1 \\ 2 & 2 & 5 - 3 \end{pmatrix}\begin{pmatrix} x \\ y \\ z \end{pmatrix} & = \begin{pmatrix} 0 \\ 0 \\ 0 \end{pmatrix}  \\
			\begin{pmatrix} -3 & -2 & -3 \\ -1 & -2 & -1 \\ 2 & 2 & 2 \end{pmatrix}\begin{pmatrix} x \\ y \\ z \end{pmatrix}           & = \begin{pmatrix} 0 \\ 0 \\ 0 \end{pmatrix}  \\
			\begin{pmatrix} -3 & -2 & -3 \\ 0 & 4 & 0 \\ 0 & 0 & 0 \end{pmatrix}\begin{pmatrix} x \\ y \\ z \end{pmatrix}              & = \begin{pmatrix} 0 \\ 0 \\ 0 \end{pmatrix}  \\
			v_1                                                                                                                        & = \begin{pmatrix} -1 \\ 0 \\ 1 \end{pmatrix} \\
		\end{split}
	\]
	Calculating the eigenvectors for $\lambda_2 = 2$:
	\[
		\begin{split}
			\begin{pmatrix} 0 - 2 & -2 & -3 \\ -1 & 1 - 2 & -1 \\ 2 & 2 & 5 - 2 \end{pmatrix}\begin{pmatrix} x \\ y \\ z \end{pmatrix} & = \begin{pmatrix} 0 \\ 0 \\ 0 \end{pmatrix}  \\
			\begin{pmatrix} -2 & -2 & -3 \\ -1 & -1 & -1 \\ 2 & 2 & 3 \end{pmatrix}\begin{pmatrix} x \\ y \\ z \end{pmatrix}           & = \begin{pmatrix} 0 \\ 0 \\ 0 \end{pmatrix}  \\
			\begin{pmatrix} -2 & -2 & -3 \\ 0 & 0 & -1 \\ 0 & 0 & 0 \end{pmatrix}\begin{pmatrix} x \\ y \\ z \end{pmatrix}             & = \begin{pmatrix} 0 \\ 0 \\ 0 \end{pmatrix}  \\
			v_2                                                                                                                        & = \begin{pmatrix} -1 \\ 1 \\ 0 \end{pmatrix}
		\end{split}
	\]
	Calculating the eigenvectors for $\lambda_3 = 1$:
	\[
		\begin{split}
			\begin{pmatrix} 0 - 1 & -2 & -3 \\ -1 & 1 - 1 & -1 \\ 2 & 2 & 5 - 1 \end{pmatrix}\begin{pmatrix} x \\ y \\ z \end{pmatrix} & = \begin{pmatrix} 0 \\ 0 \\ 0 \end{pmatrix}  \\
			\begin{pmatrix} -1 & -2 & -3 \\ -1 & 0 & -1 \\ 2 & 2 & 4 \end{pmatrix}\begin{pmatrix} x \\ y \\ z \end{pmatrix}            & = \begin{pmatrix} 0 \\ 0 \\ 0 \end{pmatrix}  \\
			\begin{pmatrix} -2 & -2 & -3 \\ 0 & 0 & -1 \\ 0 & 0 & 0 \end{pmatrix}\begin{pmatrix} x \\ y \\ z \end{pmatrix}             & = \begin{pmatrix} 0 \\ 0 \\ 0 \end{pmatrix}  \\
			v_3                                                                                                                        & = \begin{pmatrix} -1 \\ 1 \\ 1 \end{pmatrix}
		\end{split}
	\]
	The basis of eigenvectors for $A$ is $\left\{ \begin{pmatrix} -1 \\ 0 \\ 1 \end{pmatrix}, \begin{pmatrix} -1 \\ 1 \\ 0 \end{pmatrix}, \begin{pmatrix} -1 \\ 1 \\ 1 \end{pmatrix} \right\}$.
	\[
		\begin{split}
			Q        & = \begin{pmatrix} -1 & -1 & -1 \\ 0 & 1 & 1 \\ 1 & 0 & 1 \end{pmatrix}   \\
			Q^{-1}   & = \begin{pmatrix} -1 & -1 & -1 \\ -1 & 0 & -1 \\ 1 & 1 & 2 \end{pmatrix} \\
			Q^{-1}AQ & = \begin{pmatrix} 1 & 0 & 0 \\ 0 & 2 & 0 \\ 0 & 0 & 3 \end{pmatrix}
		\end{split}
	\]
\end{homeworkProblem}

% Homework problem 3 sec 5.1 problem 5i
\begin{homeworkProblem}
	For each linear operator $T$ on $V$, find the eigenvalues of $T$ and an ordered basis $\beta$ for $V$ such that $[T]_{\beta}$ is a diagonal matrix.
	\[
		V = M_{2 \times 2}(\mathbb{R}), \text{ and } T \begin{pmatrix} a & b \\ c & d \end{pmatrix} = \begin{pmatrix} c & d \\ a & b \end{pmatrix}
	\]
	\[
		\begin{split}
			\beta                                        & = \left \{ \begin{pmatrix} 1 & 0 \\ 0 & 0 \end{pmatrix}, \begin{pmatrix} 0 & 1 \\ 0 & 0 \end{pmatrix}, \begin{pmatrix} 0 & 0 \\ 1 & 0 \end{pmatrix}, \begin{pmatrix} 0 & 0 \\ 0 & 1 \end{pmatrix} \right \}         \\
			\begin{pmatrix} 1 & 0 \\ 0 & 0 \end{pmatrix} & = x_1 \begin{pmatrix} 1 & 0 \\ 0 & 0 \end{pmatrix} + x_2 \begin{pmatrix} 0 & 1 \\ 0 & 0 \end{pmatrix} + x_3 \begin{pmatrix} 0 & 0 \\ 1 & 0 \end{pmatrix} + x_4 \begin{pmatrix} 0 & 0 \\ 0 & 1 \end{pmatrix}         \\
			\beta                                        & = \left \{ \begin{pmatrix} 0 \\ 0 \\ 1 \\ 0 \end{pmatrix}, \begin{pmatrix} 0 \\ 0 \\ 0 \\ 1 \end{pmatrix}, \begin{pmatrix} 1 \\ 0 \\ 0 \\ 0 \end{pmatrix}, \begin{pmatrix} 0 \\ 1 \\ 0 \\ 0 \end{pmatrix} \right \} \\
			[T]_\beta                                    & = \begin{pmatrix} 0 & 0 & 1 & 0 \\ 0 & 0 & 0 & 1 \\ 1 & 0 & 0 & 0 \\ 0 & 1 & 0 & 0 \end{pmatrix}
		\end{split}
	\]
	Calculating the eigenvalues of $T$:
	\[
		\begin{split}
			\text{det}(T - \lambda I) & = \text{det}\left(\begin{pmatrix} 0 - \lambda & 0 & 1 & 0 \\ 0 & 0 - \lambda & 0 & 1 \\ 1 & 0 & 0 - \lambda & 0 \\ 0 & 1 & 0 & 0 - \lambda \end{pmatrix}\right) \\
			                          & = (1 - \lambda)^2(\lambda - 1)^2 = 0                                                                                                                            \\
		\end{split}
	\]
	Calculating the eigenvectors for $\lambda_1 = 1$:
	\[
		\begin{split}
			\begin{pmatrix} 0 - 1 & 0 & 1 & 0 \\ 0 & 0 - 1 & 0 & 1 \\ 1 & 0 & 0 - 1 & 0 \\ 0 & 1 & 0 & 0 - 1 \end{pmatrix}\begin{pmatrix} x \\ y \\ z \\ w \end{pmatrix} & = \begin{pmatrix} 0 \\ 0 \\ 0 \\ 0 \end{pmatrix} \\
			\begin{pmatrix} -1 & 0 & 1 & 0 \\ 0 & -1 & 0 & 1 \\ 1 & 0 & -1 & 0 \\ 0 & 1 & 0 & -1 \end{pmatrix}\begin{pmatrix} x \\ y \\ z \\ w \end{pmatrix}             & = \begin{pmatrix} 0 \\ 0 \\ 0 \\ 0 \end{pmatrix} \\
			v = a\begin{pmatrix} 1 \\ 0 \\ 1 \\ 0 \end{pmatrix} + b\begin{pmatrix} 0 \\ 1 \\ 0 \\ 1 \end{pmatrix}
		\end{split}
	\]
	Calculating the eigenvectors for $\lambda_2 = 1$:
	\[
		\begin{split}
			\begin{pmatrix} 0 - 1 & 0 & 1 & 0 \\ 0 & 0 - 1 & 0 & 1 \\ 1 & 0 & 0 - 1 & 0 \\ 0 & 1 & 0 & 0 - 1 \end{pmatrix}\begin{pmatrix} x \\ y \\ z \\ w \end{pmatrix} & = \begin{pmatrix} 0 \\ 0 \\ 0 \\ 0 \end{pmatrix} \\
			\begin{pmatrix} -1 & 0 & 1 & 0 \\ 0 & -1 & 0 & 1 \\ 1 & 0 & -1 & 0 \\ 0 & 1 & 0 & -1 \end{pmatrix}\begin{pmatrix} x \\ y \\ z \\ w \end{pmatrix}             & = \begin{pmatrix} 0 \\ 0 \\ 0 \\ 0 \end{pmatrix} \\
			v = a\begin{pmatrix} -1 \\ 0 \\ 1 \\ 0 \end{pmatrix} + b\begin{pmatrix} 0 \\ -1 \\ 0 \\ 1 \end{pmatrix}
		\end{split}
	\]
	V is a basis of eigenvectors for $T$. $V = \begin{pmatrix} 1 & 0 & 0 & 0 \\ 0 & 1 & 0 & 0 \\ 0 & 0 & 1 & 0 \\ 0 & 0 & 0 & 1 \end{pmatrix}$.

\end{homeworkProblem}

% Homework problem 4 sec 5.1 problem 11
\begin{homeworkProblem}
	Let $V$ be a finite-dimensional vector space, and let $\lambda$ be any scalar.
	\begin{enumerate}
		\item For any ordered basis $\beta$ for $V$, prove that $[\lambda|_{V}]_{\beta} = \lambda I$.
		\item Compute the characteristic polynomial of $\lambda|_{V}$.
		\item Show that $\lambda|_{V}$ is diagonalizable and has only one eigenvalue.
	\end{enumerate}
\end{homeworkProblem}

% Homework problem 5 sec 5.1 problem 22
\begin{homeworkProblem}
	Let $T$ be a linear operator on a vector space $V$ over the field $F$, and let $g(t)$ be a polynomial with coefficients form $F$. Prove that
	if $x$ is an eigenvector of $T$ with corresponding eigenvalue $\lambda$, then $g(T)(x) = g(\lambda)x$. That is, x is an eigenvector of $g(T)$ with corresponding
	eigenvalue $g(\lambda)$.
\end{homeworkProblem}

% Homework problem 6 sec 5.2 problem 2g
\begin{homeworkProblem}
	For each of the following matrices $A \in M_{n \times n}(\mathbb{R})$, test $A$ for diagonalizability, and if $A$ is diagonalizable, find an invertible matrix
	$Q$ and a diagonal matrix $D$ such that $Q^{-1}AQ = D$.
	\[
		A = \begin{pmatrix} 3 & 1 & 1 \\ 2 & 4 & 2 \\ -1 & -1 & 1 \end{pmatrix}
	\]
\end{homeworkProblem}

% Homework problem 7 sec 5.2 problem 7
\begin{homeworkProblem}
	For
	\[
		A = \begin{pmatrix} 1 & 4 \\ 2 & 3 \end{pmatrix} \in M_{2 \times 2}(\mathbb{R}),
	\]
	find a expression for $A^n$, where $n$ is an arbitrary positive integer.
\end{homeworkProblem}

% Homework problem 8 sec 5.2 problem 9
\begin{homeworkProblem}
	Let $T$ be a linear operator on a finite-dimensional vector space $V$, and suppose there exists an ordered basis $\beta$ for $V$ such that $[T]_{\beta}$ is
	an upper triangular matrix.
	\begin{enumerate}
		\item Prove that the caracteristic polynomial for $T$ splits.
		\item State and prove an analogous result for matrices.
	\end{enumerate}
	The converse of (a) is treated in Exercise 12(b).
\end{homeworkProblem}

% Homework problem 9 sec 5.4 problem 2d
\begin{homeworkProblem}
	For each of the following linear operators $T$ on the vector space $V$, determine whether the given subspace $W$ is a $T$-invariant subspace of $V$.
	\[
		V = C([0, 1]), T(f(t)) = \left[\int_{0}^{1} f(x)dt\right]t, \text{ and } W = \{f \in V : f(t) = at + b \text{ for some } a, b \}
	\]
\end{homeworkProblem}

% Homework problem 10 sec 5.4 problem 6d
\begin{homeworkProblem}
	For each linear operator $T$ on the vector space $V$, find an ordered basis for the $T$-cyclic subspace generated by the given vector $z$
	\[
		V = M_{2 \times 2}(\mathbb{R}), T(A) = \begin{pmatrix} 1 & 1 \\ 2 & 2 \end{pmatrix}A, \text{ and } z = \begin{pmatrix} 0 & 1 \\ 1 & 0 \end{pmatrix}
	\]
\end{homeworkProblem}

% Homework problem 11 sec 5.4 problem 10d
\begin{homeworkProblem}
	For each linear operator in Exercise 6, find the characteristic polynomial of $f(t)$ of $T$, and verify that the characteristic polynomial of $T_{W}$
	(computed in Exercise 6) divides $f(t)$.
\end{homeworkProblem}

% Homework problem 12 sec 5.4 problem 11
\begin{homeworkProblem}
	Let $T$ be a linear operator on a vector space $V$, let $v$ be a nonzero vector in $V$, and let $W$ be the $T$-cyclic subspace of $V$ generated by $v$.
	Prove that
	\begin{enumerate}
		\item $W$ is $T$-invariant.
		\item Any $T$-invariant subspace of $V$ containing $v$ also contains $W$.
	\end{enumerate}
\end{homeworkProblem}

% Homework problem 13 sec 6.1 problem 3
\begin{homeworkProblem}
	In $C([0, 1])$, let $f(t) = t$ and $g(t) = e^{t}$. Compute $\langle f, g \rangle$. (as defined in Example 3), $\|f\|$, $\|g\|$, and $\|f + g\|$. Then verify
	both the Cauchy-Schwarz inequality and the triangle inequality.
\end{homeworkProblem}

% Homework problem 14 sec 6.1 problem 5
\begin{homeworkProblem}
	In $C^2$, show that $\langle x, y \rangle = xAy^{\star}$ is an inner product, where
	\[
		A = \begin{pmatrix} 1 & i \\ -i & 2 \end{pmatrix}
	\]
	Compute $\langle x, y \rangle$ for $x = (1 - i, 2 + 3i)$ and $y = (2 + i, 3 -2i)$.
\end{homeworkProblem}

% Homework problem 15 sec 6.1 problem 8b
\begin{homeworkProblem}
	Provide reasons why each of the following is not an inner product on the given vector space.
	\[
		\langle A, B \rangle + tr(A + B) \text{ on } M_{2 \times 2}(\mathbb{R})
	\]
\end{homeworkProblem}

% Homework problem 16 sec 6.1 problem 13
\begin{homeworkProblem}
	Suppose that $\langle \cdot, \cdot \rangle_1$ and $\langle \cdot, \cdot \rangle_2$ are two inner products on a vector space $V$. Prove that
	$\langle \cdot, \cdot \rangle = \langle \cdot, \cdot \rangle_1 + \langle \cdot, \cdot \rangle_2$ is another inner product on $V$.
\end{homeworkProblem}

% Homework problem 17 sec 6.1 problem 17
\begin{homeworkProblem}
	Let $T$ be a linear operator on an inner product space $V$, and suppose $\|T(x)\| = \|x\|$ for all $x$. Prove that $T$ is one-to-one.
\end{homeworkProblem}

% Homework problem 18 sec 6.1 problem 26c
\begin{homeworkProblem}
	Prove that the following are norms on the given vector space $V$.
	\[
		V = C([0,1]), \|f\|_{V} = \int_{0}^{1} |f(t)|dt \text{ for all } f \in V
	\]
\end{homeworkProblem}

% Homework problem 19 sec 6.2 problem 2c
\begin{homeworkProblem}
	In each part, apply the Gram-Schmidt process to the given subset $S$ of the inner product space $V$ to obtain an orthogonal basis for
	$\text{span}(S)$. Then normalize the vectors in this basis to obtain an orthonormal basis $\beta$ for $\text{span}(S)$, and compute
	the Fourier coefficients of the given vector relative to $\beta$. Finally, use Theorem 6.5 to verify your results.
	\[
		V = P_2(\mathbb{R}), \text{ with the inner product } \langle f(x), g(x) \rangle = \int_{0}^{1} f(t)g(t)dt,
		S = \{1, x, x^2\}, \text{ and } h(x) = 1 + x
	\]
\end{homeworkProblem}

% Homework problem 20 sec 6.2 problem 2i
\begin{homeworkProblem}
	Same as the previous problem
	\[
		V + span(S) \text{ with the inner product } \langle f(x), g(x) \rangle = \int_{0}^{\pi} f(t)g(t)dt,
		S = \{\sin(t), \cos(t), 1, t\}, \text{ and } h(x) = 2t + 1
	\]
\end{homeworkProblem}

% Homework problem 21 sec 6.2 problem 4
\begin{homeworkProblem}
	Let $S = \{(1,0,i), (1,2,1)\}$ in $C^3$. Compute $S^{\perp}$.
\end{homeworkProblem}

% Homework problem 22 sec 6.2 problem 9
\begin{homeworkProblem}
	Let $W = span(\{(i,0,1) \})$ in $C^3$. Find orthogonal bases for $W$ and $W^{\perp}$.
\end{homeworkProblem}

% Homework problem 23 sec 6.2 problem 19a
\begin{homeworkProblem}
	In each of the following parts, find the orthogonal projection of the given vector on the given subspace $W$ of the inner product space
	$V$.
	\[
		V = R^2, u = (2,6), \text{ and } W = \{(x,y): y = 4x\}
	\]
\end{homeworkProblem}

\end{document}
