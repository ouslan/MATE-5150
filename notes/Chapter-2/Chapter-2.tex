\documentclass[10pt, oneside]{article}
\usepackage{amsmath, amsthm, amssymb, calrsfs, wasysym, verbatim, bbm, color, graphics, geometry}

\geometry{tmargin=.75in, bmargin=.75in, lmargin=.75in, rmargin = .75in}

\newcommand{\R}{\mathbb{R}}
\newcommand{\Rmn}{\mathbb{R}_{m \times n}}
\newcommand{\C}{\mathbb{C}}
\newcommand{\Z}{\mathbb{Z}}
\newcommand{\N}{\mathbb{N}}
\newcommand{\Q}{\mathbb{Q}}
\newcommand{\nulla}{N(A^T)}
\newcommand{\ranka}{r(A)}
\newcommand{\Cdot}{\boldsymbol{\cdot}}

\newtheorem{thm}{Theorem}
\newtheorem{defn}{Definition}
\newtheorem{conv}{Convention}
\newtheorem{rem}{Remark}
\newtheorem{lem}{Lemma}
\newtheorem{cor}{Corollary}


\title{MATE 5150: Teoría de Espacios Vectoriales}
\author{Alejandro M. Ouslan}
\date{}

\begin{document}

\maketitle
\tableofcontents

\vspace{.25in}

\section{Los Cuatro Subespacios Fundamentales Parte I}

\section{Los Cuatro Subespacios Fundamentales Parte II}
\subsection{Espacio Nullo}
\begin{enumerate}
	\item $A \in \Rmn$
	\item $ N(A) = \{ z \in \R^n : Az =0 \}$
	\item Base: gen $ \{ U_1, U_2, \cdots, U_i $
\end{enumerate}
\subsection{Dimension:}
\begin{enumerate}
	\item Fila: $ E(A_T):$
\end{enumerate}

\subsection{}

\rem{
	Sea $ A \in \Rmn$ y $U$ una matiz esclanodada y una base para $R(A)$ (es sun sub de $R^m$) esta
	formada por las columnas de $A$ corespodientes a las columnas de $U$ que tienen pivote:
	$\dim  R(A) = r(A)$
}

\[
	\begin{bmatrix}
		0  & 0  & 0  & 1 \\
		1  & -1 & -2 & 0 \\
		-1 & 1  & 2  & 1 \\
	\end{bmatrix}
\]
\[
	P_{12}, f_1 + f_3, -f_2+f_3
\]
\[
	U = \begin{bmatrix}
		1 & -1 & -2 & 0 \\
		0 & 0  & 0  & 1 \\
		0 & 0  & 0  & 0
	\end{bmatrix}
\]
base $ R(A) = \{ \begin{bmatrix}0 \\ 1 \\ -1\end{bmatrix}, \begin{bmatrix} 1 \\ 0 \\ 1
	\end{bmatrix} $

\subsection{Base y Dimension Nulo Izquierdo}
$N(A^T) = \{ Z \in \R^m : A ^TZ = 0\}$

\rem{
	Sea $A\in Rmn$ tal que $r9A)=r$, P una matriz permutacion, L una matriz inferor con dagonal unitaria., y U
	una mateiz escalonada tal que $PA = LU$.
	\begin{enumerate}
		\item dim $ N(A^T) = M-r(A)$
		\item Base $\nulla$. esta data por los vectores que U parecen como los ultimos $(m-r)$
		      filas de las matriz $L^{-1)P}$
	\end{enumerate}
}

\subsectio
\end{document}
